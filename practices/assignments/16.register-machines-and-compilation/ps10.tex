% Master File: ps10.tex
% Latex Macro package for 6.001
% done by Nikhil, November, 1988
% updated by Hal, spring 1988
% updated by Arthur, fall 1989

\documentstyle[11pt]{article}

\pagestyle{myheadings}

% ALIGN EVEN- AND ODD-NUMBERED PAGES.
\evensidemargin 35pt

% NO NUMBERING ON SECTIONS
\setcounter{secnumdepth}{0}

% HORIZONTAL MARGINS
% Left margin 1 inch (0 + 1)
\setlength{\oddsidemargin}{0in}
% Text width 6.5 inch (so right margin 1 inch).
\setlength{\textwidth}{6.5in}

% ----------------
% VERTICAL MARGINS
% Top margin 0.5 inch (-0.5 + 1)
\setlength{\topmargin}{-0.5in}
% Head height 0.25 inch (where page headers go)
\setlength{\headheight}{0.25in}
% Head separation 0.25 inch (between header and top line of text)
\setlength{\headsep}{0.25in}
% Text height 8.5 inch (so bottom margin 1.5 in)
\setlength{\textheight}{8.5in}

% ----------------
% PARAGRAPH INDENTATION
\setlength{\parindent}{0in}

% SPACE BETWEEN PARAGRAPHS
\setlength{\parskip}{\medskipamount}

% ----------------
% EVALUATION SYMBOL
\newcommand{\evalsto}{$\Longrightarrow$}

% ----------------
% STRUTS
% HORIZONTAL STRUT.  One argument (width).
\newcommand{\hstrut}[1]{\hspace*{#1}}
% VERTICAL STRUT. Two arguments (offset from baseline, height).
\newcommand{\vstrut}[2]{\rule[#1]{0in}{#2}}

% ----------------
% EMPTY BOXES OF VARIOUS WIDTHS, FOR INDENTATION
\newcommand{\hm}{\hspace*{1em}}
\newcommand{\hmm}{\hspace*{2em}}
\newcommand{\hmmm}{\hspace*{3em}}
\newcommand{\hmmmm}{\hspace*{4em}}

% ----------------
% VARIOUS CONVENIENT WIDTHS RELATIVE TO THE TEXT WIDTH, FOR BOXES.
\newlength{\hlessmm}
\setlength{\hlessmm}{\textwidth}
\addtolength{\hlessmm}{-2em}

\newlength{\hlessmmmm}
\setlength{\hlessmmmm}{\textwidth}
\addtolength{\hlessmmmm}{-4em}

% ----------------
% ``TIGHTLIST'' ENVIRONMENT (no para space between items, small indent)
\newenvironment{tightlist}%
{\begin{list}{$\bullet$}{%
    \setlength{\topsep}{0in}
    \setlength{\partopsep}{0in}
    \setlength{\itemsep}{0in}
    \setlength{\parsep}{0in}
    \setlength{\leftmargin}{1.5em}
    \setlength{\rightmargin}{0in}
    \setlength{\itemindent}{0in}
}
}%
{\end{list}
}

% ----------------
% CODE FONT (e.g. {\cf x := 0}).
\newcommand{\cf}{\footnotesize\tt}

% ----------------
% INSTRUCTION POINTER
\newcommand{\IP}{$\bullet$}
\newcommand{\goesto}{$\longrightarrow$}

% ----------------------------------------------------------------
% LISP CODE DISPLAYS.
% Lisp code displays are enclosed between \bid and \eid.
% Most characters are taken verbatim, in typewriter font,
% Except:
%  Commands are still available (beginning with \)
%  Math mode is still available (beginning with $)

\outer\def\beginlisp{%
  \begin{minipage}[t]{\linewidth}
  \begin{list}{$\bullet$}{%
    \setlength{\topsep}{0in}
    \setlength{\partopsep}{0in}
    \setlength{\itemsep}{0in}
    \setlength{\parsep}{0in}
    \setlength{\leftmargin}{1.5em}
    \setlength{\rightmargin}{0in}
    \setlength{\itemindent}{0in}
  }\item[]
  \obeyspaces
  \obeylines \footnotesize\tt}

\outer\def\endlisp{%
  \end{list}
  \end{minipage}
  }

{\obeyspaces\gdef {\ }}

% ----------------
% ILLUSTRATIONS
% This command should specify a NEWT directory for ps files for illustrations.
\def\psfileprefix{/usr/nikhil/parle/}
\def\illustration#1#2{
\vbox to #2{\vfill\special{psfile=\psfileprefix#1.ps hoffset=-72 voffset=-45}}} 

% \illuswidth is used to set up boxes around illustrations.
\newlength{\illuswidth}
\setlength{\illuswidth}{\textwidth}
\addtolength{\illuswidth}{-7pt}

% ----------------------------------------------------------------
% SCHEME CLOSURES AND PROCEDURES

% CLOSURES: TWO CIRCLES BESIDE EACH OTHER; LEFT ONE POINTS DOWN TO CODE (arg 1)
% RIGHT ONE POINTS RIGHT TO ENVIRONMENT (arg 2)
\newcommand{\closure}[2]{%
\begin{tabular}[t]{l}
\raisebox{-1.5ex}{%
  \setlength{\unitlength}{0.2ex}
  \begin{picture}(25,15)(0,-7)
   \put( 5,5){\circle{10}}
   \put( 5,5){\circle*{1}}
   \put( 5,5){\vector(0,-1){10}}
   \put(15,5){\circle{10}}
   \put(15,5){\circle*{1}}
   \put(15,5){\vector(1,0){12}}
  \end{picture}}
  \fbox{\footnotesize #2} \\
%
\hspace*{0.8ex} \fbox{\footnotesize #1}
\end{tabular}
}

% PROCEDURES: BOX CONTAINING PARAMETERS (arg 1) AND BODY (arg 2)
\newcommand{\proc}[2]{%
\begin{tabular}{l}
params: #1 \\
body: #2 \\
\end{tabular}
}

% PROBLEM SET HEADER -- args are semester and problem set or solution
% example: \psetheader{Spring Semester, 1989}{Problem set 1}
\newcommand{\psetheader}[2]{%
\markright{6.001, #1---#2}
\begin{center}
MASSACHVSETTS INSTITVTE OF TECHNOLOGY \\
Department of Electrical Engineering and Computer Science \\
6.001---Structure and Interpretation of Computer Programs \\
#1 \\
\medskip
{\bf #2}
\end{center}
}

% PROBLEM HEADER
\newcommand{\problem}[1]{{\bf #1}}

% KEYS
\newcommand{\key}[1]{\fbox{{\sc #1}}}
\newcommand{\ctrl}{\key{ctrl}--}
\newcommand{\shift}{\key{shift}--}
\newcommand{\run}{\key{run} \ }
\newcommand{\runkey}[1]{\run \key{#1}}
\newcommand{\extend}{\key{extend} \ }
\newcommand{\kkey}[1]{\key{k$_{#1}$}}

%% Examples of keys
%% \key{abort}
%% \ctrl\key{g}
%% \extend\key{logout}
%% \kkey{1}
%% \shift\kkey{1}

% ----------------------------------------------------------------
% HERE BEGINS THE DOCUMENT
% start with \begin{document}


\newcommand{\Code}[1]{\mbox{\tt #1}}

\begin{document}

\psetheader{Sample Problem Set}{Compilation}
\begin{center}
{\bf Register Machines and Compilation}
\end{center}

\medskip

Register machines provide a means of customizing code for particular
processes.  In principle, customization leads to more efficient code,
since one can avoid the overhead that comes from a compiler's obligation
to handle more general computations\footnote{For example, the compiler in Chapter
5 generates code in which arguments of procedures are maintained as a list
in the {\tt argl} register.  On the other hand, a register machine
customized for a procedure of, say, three arguments might usefully keep
the arguments in three separate registers.}.
In this problem set you will handcraft two simple machines and compare
them to the compiler in Chapter 5 of the notes.

The register machines you define should use only the following few
primitives:
\beginlisp
+ - * / inc dec = < > zero?  not true false
cons car cdr pair? null? list eq? symbol? write-line
\endlisp
Even though code generated by the compilers uses more complex primitives
such as
\beginlisp
lookup-variable-value extend-environment \ldots,
\endlisp
your hand-crafted code should turn out to run more efficiently.

Remember that register machine instructions are only of the
following types:
\beginlisp
test branch assign goto save restore perform
\endlisp
In this problem set, you should not need to use {\tt perform}.
Remember also that the only values that can be assigned to registers
or tested in branches are constants, fetches from registers, or
primitive operations applied to fetches from registers.  No nested
operations are permitted\footnote{For example,
(assign val (op inc) ((op *) (reg a) (reg b)))
is {\em not} permitted, since a call to {\tt *} is nested inside a
call to {\tt inc}.}.

\medskip
Here are the definitions of two Scheme procedures for deleting all
occurrences of an element, {\tt x}, from a list, {\tt l}:
\beginlisp
(define (delq1 x l)
  (cond ((null? l) '())
        ((eq? x (car l)) (delq1 x (cdr l)))
        (else (cons (car l) (delq1 x (cdr l))))))
\endlisp

\medskip
\beginlisp
(define (delq2 x l)
  (define (delete-reverse maybe-xs no-xs)
    (cond ((null? maybe-xs) no-xs)
          ((eq? x (car maybe-xs)) (delete-reverse (cdr maybe-xs) no-xs))
          (else (delete-reverse (cdr maybe-xs) (cons (car maybe-xs) no-xs)))))
  (delete-reverse (delete-reverse l '()) '()))
\endlisp

\paragraph{PreLab exercise 1A:}
For each of these procedures, say what order of growth in time (total
machine operations) and space (maximum stack depth) you expect them to use.

\paragraph{PreLab exercise 1B:}
Implement both of these procedures as register machines.  You should show
the controllers for all the machines, but you need show the data paths for
only one of them.

\section{To do in lab}

In lab, you will use the register machine simulator to test the register
machines you designed in exercise 1B.  To use the simulator, load the code
for problem set 10 and type in your machine definitions:

\beginlisp
(define my-machine
  (make-machine
   '(x l val \ldots)
   standard-primitives          ; + - * / inc, etc.
   '((test \ldots)
     \vdots
     )))
\endlisp

You'll find it convenient to define test procedures that load an input
into a machine, run the machine, print some statistics, and return the
result computed by the machine.  For example:

\beginlisp
(define (test-machine x l)
  (set-register-contents! my-machine 'x x)
  (set-register-contents! my-machine 'l l)
  (my-machine 'initialize-stack)
  (my-machine 'initialize-ops-counter)
  (start my-machine)
  (my-machine 'print-stack-statistics)
  (my-machine 'print-ops)
  (get-register-contents my-machine 'val))
\endlisp

In addition to routines that gather statistics for stack usage and
total number of operations, there are some procedures to help you
debug your machines.  {\tt trace-reg-on} will show all
assignments to a specified machine register as they occur. Evaluating:
\beginlisp
(trace-reg-on my-machine 'l)
\endlisp
before running your test procedure will show you all the changes to the
{\tt l} register. To see even more stuff, try:

\beginlisp
(trace-on my-machine)
\endlisp

which will print each machine instruction as it is executed.  To get rid
of these traces, use {\tt trace-reg-off} and {\tt trace-off}.

\paragraph{Lab exercise 2A:}
Debug your machines, run them on some representative inputs, and make a
table that records the total number of machine operations, total number of
stack pushes, and maximum stack depth as a function of the length of the
list from which an element is being deleted.

\paragraph{PostLab exercise 2B:}
Try to derive formulas for the total number of machine operations, total
number of pushes, and maximum stack depth used by your machines, as
functions of the length of the list.  In most cases, the functions will
turn out to be polynomials in the list length, in which case you should be
able to exhibit exact formulas, not just orders of growth.

\subsubsection{Running the Compiler}

There are two ways to run the compiler.  First, you may simply compile an
expression and obtain the list of machine instructions as a result, so
that you can study it.  For instance,

\beginlisp
(define test-expression '(define (f x y) (* (+ x y) (- x y))))

(define result (compile test-expression 'val 'return))

(pp result)
\endlisp

A way to look just at the produced code even more easily is:

\beginlisp
(compile-and-display test-expression)
\endlisp
which does exactly the same call to {\tt compile} as above.

The second way to run the compiler is to apply the procedure {\tt
compile-and-go} to the expression.  This compiles the expression and
executes it in the environment of the explicit control evaluator machine
{\tt eceval}.  When evaluation is complete, you are left in the
read-eval-print loop talking to the explicit control evaluator.  Then you
can experiment with the compiled expression by evaluating further
expressions.

\subsubsection{Running the Evaluator}

The evaluator for this problem set is the explicit control evaluator of
section 5.4.  For your convenience, we have extended it to handle {\tt
cond} and {\tt let}.

To evaluate an expression in the {\tt eceval} read-eval-print loop,
type the expression after the prompt, followed by {\tt ctrl-X ctrl-E}.
After each evaluation, the simulator will print the number of stack
and machine operations required to execute the code.\footnote{These
counts may include a few extra operations needed to run the driver
loop itself.  This is a small constant overhead that you can ignore
when you collect statistics.}

Here is an example:

\beginlisp
(compile-and-go
 '(define (fact n) (if (= n 0) 1 (* n (fact (- n 1)))))))
(total-pushes = 0 maximum-depth = 0)
(machine-ops = 12)
;;; EC-Eval value:  (the-unspecified-value)
\endlisp

\beginlisp
;;; EC-Eval input: (fact 4)          <== you type this and ctrl-X ctrl-E
(total-pushes = 31 maximum-depth = 14)
(machine-ops = 278)
;;; EC-Eval value: 24
\endlisp

\beginlisp
;;; EC-Eval input: (fact (fact 3))   <== you type this
(total-pushes = 68 maximum-depth = 20)
(machine-ops = 594)
;;; EC-Eval value: 720
\endlisp

\beginlisp
;;; EC-Eval input: fact
(total-pushes = 0 maximum-depth = 0)
(machine-ops = 13)
;;; EC-Eval value: <compiled-procedure>
\endlisp

\beginlisp
;;; EC-Eval input:
(define (fact n) (if (= n 0) 1 (* n (fact (- n 1))))) ;<== fact gets redefined
(total-pushes = 3 maximum-depth = 3)
(machine-ops = 45)
;;; EC-Eval value: (the-unspecified-value)
\endlisp

\beginlisp
;;; EC-Eval input: fact
(total-pushes = 0 maximum-depth = 0)
(machine-ops = 13)
;;; EC-Eval value:
;;;  (compound-procedure (n)
;;;    ((if (= n 0) 1 (* n (fact (- n 1))))) <procedure-env>)
\endlisp

\beginlisp
;;; EC-Eval input:
(fact 4)                ;<== redefined fact gets interpreted -- slower!
(total-pushes = 144 maximum-depth = 20)
(machine-ops = 1572)
;;; EC-Eval value: 24
\endlisp

To exit back to regular Scheme type {\tt ctrl-C ctrl-C}.  To reenter the
evaluator with the previous global environment, you may do another {\tt
compile-and-go}, or you may simply evaluate {\tt (eval-loop)} in Scheme.
To start the evaluator with a reinitialized global environment, evaluate
{\tt start-eceval}.


\paragraph{Lab exercise 3A:}
Compile and run the (Scheme) definitions of the {\tt delq1} and {\tt
delq2} procedures, and make tables to record statistics.

\paragraph{Lab exercise 3B:}
Now redefine them within the {\tt eceval} read-eval-print loop and record
corresponding statistics for the interpreted definitions.

\paragraph{PostLab exercise 3C:}
Derive formulas for the total number of machine operations, total number
of pushes, and maximum stack depth required, as functions of the length of
the list, for the compiled and interpreted delete procedures.

\bigskip

The file {\tt naivecom.scm} contains the simple code generator
described in lecture which omits the stack optimizations carried out
in the Notes.  Because its procedures satisfy a simplified
``contract'' slightly different from that in the Notes, the naive
compiler is not compatible with the evaluator.  We display its code
using {\tt naive-compile-and-display}, and run naive code using {\tt
naive-compile-and-go}.  For example: \beginlisp (define (fibexp n)
  `(begin
     (define (fib m)
       (if (< m 2)
           m
           (+ (fib (- m 1)) (fib (- m 2)))))
     (fib ,n)))
\endlisp

\beginlisp
(naive-compile-and-go (fibexp 8))
(total-pushes = 1469 maximum-depth = 27)
(machine-ops = 6544)
;value of expression: 21
;Value: \#[useless-value]
\endlisp

Compare this to:
\beginlisp
(compile-and-go (fibexp 8))
(total-pushes = 332 maximum-depth = 23)
(machine-ops = 2772 )
;;; EC-Eval value: 21
\endlisp

So in this case, the optimized code executes about 40\% as many
instructions and uses 85\% of the stack space as the code generated by
the naive compiler.

\paragraph{Lab exercise 4:}  Repeat exercise 3 for the naively compiled
delete procedures.

\paragraph{Post Lab exercise 5:} We'll consider the time used for a
computation to be the total number of machine operations, and the space
used to be the maximum stack depth.  For each of your list-deletion
procedures, determine the limiting ratio, as the list length becomes
large, of the time and space requirements for your hand-coded machines,
versus the time and space requirements for the compiled, naive-compiled,
and interpreted code.


\paragraph{Lab exercise 6A:} Make listings of the code generated by the
compiler from the Notes for the definitions of {\tt delq1} and {\tt
delq2}.

\paragraph{PostLab exercise 6B:}
Compare the listings with your hand-coded versions to see why the
compiler's code is less efficient than yours.  Suggest one improvement to
the compiler that could lead it to do a better job.  Write one or two
clear paragraphs indicating how you might go about implementing your
improvement.  You needn't actually carry out the the implementation, but
your description should be reasonably precise.  For example, you should
say what new information the compiler should keep track of, what new data
structures may be required to maintain this information, and how the
information should be used in generating the new, improved code.

\end{document}

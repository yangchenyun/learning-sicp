% Document Type: LaTeX
% Master File: ps7oop.tex
\input 6001mac

\def\fbox#1{%
  \vtop{\vbox{\hrule%
     \hbox{\vrule\kern3pt%
 \vtop{\vbox{\kern3pt#1}\kern3pt}%
 \kern3pt\vrule}}% 
 \hrule}}

\begin{document}

%%%%%%%%%%%%%%%%%%%%%%%%%%%%%%%%%%%%%%%%%%%%%%%%%%%%%%%%%%%%%%%%%%%%%%%%%%%%%%%

\psetheader{Sample assignment}{Object-oriented programming}

%%%%%%%%%%%%%%%%%%%%%%%%%%%%%%%%%%%%%%%%%%%%%%%%%%%%%%%%%%%%%%%%%%%%%%%%%%%%%%%


\medskip

\begin{flushleft}
Reading: Section 4.1
\end{flushleft}

\begin{center}
{\Large\bf Languages for Object-Oriented Programming}
\end{center}

\fbox{This problem set is probably the most difficult one of the
semester, but paradoxically, the one that asks you to write the least
amount of code, and for which you should have to spend the least time
in lab, {\it provided that you prepare before you come to lab}.
Instead of asking you to do a lot of implementation, we are asking
you to assume the role of language designer, and to think about and
discuss some issues in the design of languages for object-oriented
programming.  Note especially that there is a significant part of
this problem set to be completed {\it after} you have finished in the
lab.}

\smallskip

\fbox{This problem set has been designed so that the interpreter you
will be dealing with is an extension of the metacircular evaluator in
chapter 4 of the book.  The implementation below is described with
reference to the programs in the book.  In order to understand what
is going on, it will be necessary to work through section 4.1 before
starting on this assignment.}

\medskip

Although Object-Oriented programming has become very popular, the
design of languages to support Object-Oriented programming is still an
area of active research.  In this problem set, you will be dealing
with issues that are not well-understood, and around which there are
major disagreements among language designers.  The questions in this
problem set will not ask you to supply ``right answers.''  Instead,
they will ask you to make reasonable design choices, and to be able to
defend these choices.  We hope you will appreciate that, once you have
come to grips with the notion of an interpreter, you are in a position
to address major issues in language design, even issues that are at
the forefront of current research.\footnote{This problem set was
developed by Hal Abelson, Greg McLaren, and David LaMacchia.  It draws
on a Scheme implementation of Oaklisp by McLaren and a Scheme
implementation of Dylan by Jim Miller.  The organization of the
generic function code follows the presentation of the Common Lisp
Object System (CLOS) in {\em The Art of the Metaobject Protocol}, by
Gregor Kiczales, Jim des Rivi\`eres, and Dan Bobrow (MIT Press,
1991).}


\paragraph{Tutorial exercise 1:} Do exercise 4.2 of the notes.  Don't
actually go to lab to implement this.  Just be able to explain
precisely what procedures need to be modified, what new procedures
need to be written, and what the code must do.

\section{1. Issues in object-oriented language design}

We've already seen two different approaches to implementing generic
operations.  One is {\it data-directed programming}, which relies on a
table to dispatch based on the types of arguments.  The second method,
{\it message-passing}, represents objects as procedures with local
state.  As we saw in problem set 5, these objects can be arranged in
complex {\it inheritance} relationships, such as ``a troll is a kind
of person.''

One drawback with both of these approaches is that they make a
distinction between generic operations and ordinary procedures, or
between message-passing objects and ordinary data.  This makes it
awkward, for example, to extend an ordinary procedure so that it also
works as a generic operation on new types of data.  For instance, we
might like to extend the addition operator {\tt +} so that it can add
two vectors, rather than having to define a separate {\tt vector-add}
procedure.

Recent experiments with object-oriented languages have attempted to
integrate objects into the core of the language, rather than building
an object system on top of the language.  The idea is that {\it
everything} in the language is an object, and {\it all} procedures
are generic operations.  Two such languages, both based upon Scheme,
are {\it Oaklisp}, developed in 1986 by Kevin Lang and Barak
Pearlmutter at CMU, and {\it Dylan}$^{\rm TM}$ ({\it Dy\/}namic {\it
Lan\/}guage), currently under development at the Apple Research
Center in Cambridge.  The language we will implement in this problem
set is called {\it MIT TOOL} (Tiny Object Oriented Language).  It is
essentially a (very) simplified version of Dylan, designed to make
the implementation an easy extension of the metacircular evaluator of
chapter 4.

\subsection{1.1 Classes, instances, and generic functions}

The framework we'll be using in TOOL (which is the same as in many
object-oriented systems) includes basically the same ideas as we've
already seen, although with different terminology.  An object's
behavior is defined by its {\it class\/}---the object is said to be an
{\it instance} of the class.  All instances of a class have identical
behavior, except for information held in a set of specified {\it
slots}, which provides the local state for the instance.  Following
Dylan, we'll use the convention of naming classes with names that are
enclosed in angle brackets, for example {\tt <account>} or {\tt
<number>}.\footnote{Keep in mind that this use of brackets is a naming
convention only---like naming predicates with names that end in
question mark.}

The {\tt define-class} special form creates a new kind of class.  You
specify the name of the class, the class's {\it superclass}, and the
names for the slots.  In TOOL, every class has a superclass, whose
behavior (and slots) it inherits.  There is a predefined class called
{\tt <object>} that is the most general kind of object.  Every TOOL
class has {\tt <object>} as an ancestor.  Once you have defined a
class, you use the special form {\tt make} to create instances of it.
{\tt Make} takes the class as argument, together with a list that
specifies values for the slots.  The order in which the slots and
values are listed does not matter, since each slot is identified by
name.  For example, we can specify that a ``cat'' is a kind of object
that has a size and a breed, and then create an instance of {\tt
<cat>}.  Note the use of the {\tt get-slot} procedure to obtain the
value in a designated slot.

\beginlisp
TOOL==> (define-class <cat> <object> size breed)
(defined class: <cat>)
\null
TOOL==> (define garfield (make <cat> (size 6) (breed 'weird)))
*undefined*     ;as in Scheme, define returns the undefined value
\null
TOOL==> (get-slot garfield 'breed)
weird
\endlisp

Procedures in TOOL are all {\tt generic-functions}, defined with the
special form {\tt make-generic-function}:

\beginlisp
TOOL==> (define-generic-function 4-legged?)
(defined generic function: 4-legged?)
\endlisp

\noindent You can think of a newly defined generic function as an
empty table to be filled in with {\it methods}.  You use {\tt
define-method} to specify methods for a generic function that
determine its behavior on various classes.

\beginlisp
TOOL==> (define-method 4-legged? ((x <cat>))
           true)
(added method to generic function: 4-legged?)
\null
TOOL==> (define-method 4-legged? ((x <object>))
           'Who-knows?)
(added method to generic function: 4-legged?)
\null
TOOL==> (4-legged? garfield)
\#t
TOOL==> (4-legged? 'Hal)
who-knows?
\endlisp

\noindent
The list in {\tt define-method} following the generic function name is
called the list of {\it specializers} for the method.  This is like an
argument list, except that it also specifies the class of each
argument.  In the first example above, we define a method for {\tt
4-legged?} that takes one argument named {\tt x}, where {\tt x} is a
member of the class {\tt <cat>}.  In the second example, we define
another method for {\tt 4-legged?} that takes one argument named {\tt
x}, where {\tt x} is a member of the class {\tt <object>}.  Now {\tt
4-legged?} will return true if the argument is a cat, and will return
{\tt who-knows?} if the argument is an object.  Notice that {\tt
garfield} is an object as well as a cat (because {\tt <object>} is the
superclass of {\tt <cat>}).  Yet, when we call {\tt 4-legged?} with
{\tt garfield} as an input, TOOL uses the method for {\tt <cat>}, and
not the method for {\tt <object>}.  In general, TOOL uses the {\it
most specific method} that applies to the inputs.\footnote{See the
code (below) for a definition of ``most specific method.''  This is
one of the things that language designers argue about.}

In a similar way, we can define a new generic function {\tt say} and
give it a method for cats (and subclasses of cats):

\beginlisp
TOOL==> (define-generic-function say)
(defined generic function: say)
\null
TOOL==> (define-method say ((cat <cat>) (stuff <object>))
           (print 'meow:)  ;print is TOOL's procedure for printing things
           (print stuff))
(added method to generic function: say)
\null
TOOL==> (define-class <house-cat> <cat> address)
(defined class: <house-cat>)
TOOL==> (define fluffy         ;note that a house cat is a cat, and therefore
          (make <house-cat>    ;has slots for breed and size, as well
                (size 'tiny))) ;as for address
\endlisp

\beginlisp
TOOL==> (get-slot fluffy 'breed)
*undefined*                   ;we never initialized fluffy's breed
\endlisp

\beginlisp
TOOL==> (say garfield '(feed me))
meow:
(feed me)
TOOL==> (say fluffy '(feed me))
meow:
(feed me)
TOOL==> (say 'hal '(feed me))
;No method found -- APPLY-GENERIC-FUNCTION
\endlisp

\noindent In the final example, TOOL gives an error message when we
apply {\tt say} to the symbol {\tt hal}.  This is because {\tt hal}
is a symbol (not a cat) and there is no {\tt say} method defined for
symbols.

We can go on to define a subclass of {\tt <cat>}:

\beginlisp
TOOL==> (define-class <show-cat> <cat> awards)
(defined class: <show-cat>)
\null
TOOL==> (define-method say ((cat <show-cat>) (stuff <object>))
           (print stuff)
           (print '(I am beautiful)))
(added method to generic function: say)
\null
TOOL==> (define Cornelius-Silverspoon-the-Third
           (make <show-cat>
                 (size 'large)
                 (breed '(Cornish Rex))
                 (awards '((prettiest skin)))))
*undefined*
\null
TOOL==> (say cornelius-silverspoon-the-Third '(feed me))
(feed me)
(i am beautiful)
\endlisp

\beginlisp
TOOL==> (define-method say ((cat <cat>) (stuff <number>))
          (print '(cats never discuss numbers)))
(added method to generic function: say)
\null
TOOL==> (say fluffy 37)
(cats never discuss numbers)
\endlisp

As the final example illustrates, TOOL picks the appropriate method
for a generic function by examining the classes of all the arguments
to which the function is applied.  This differs from the
message-passing model, where the dispatch is done by a single object.

Notice also that TOOL knows that 37 is a member of the class {\tt
<number>}.  In TOOL, {\it every} data object is a member of some
class.  The classes {\tt <number>}, {\tt <symbol>}, {\tt <list>}, and
{\tt <procedure>} are predefined, with {\tt <object>} as their
superclass.  Also, {\it every} procedure is a generic procedure,
to which you can add new methods.  The following generic procedures
are predefined, each initially with a single method as indicated
by the specializer:

\beginlisp
+         (<number> <number>)
-         (<number> <number>)
*         (<number> <number>)
/         (<number> <number>)
=         (<number> <number>)
>         (<number> <number>)
<         (<number> <number>)
sqrt      (<number>)
cons      (<object> <object>)
append    (<list> <list>)
car       (<list>)
cdr       (<list>)
null?     (<object>)
print     (<object>)
get-slot  (<object> <symbol>)
set-slot! (<object> <symbol> <object>)
\endlisp

\paragraph{Tutorial exercise 2:} Show how to implement
two-dimensional vector arithmetic in TOOL by extending the generic
functions {\tt +} and {\tt *}, which are already predefined to work
on numbers.  Define a class {\tt <vector>} with slots {\tt xcor} and
{\tt ycor}.  Arithmetic should be defined so that adding two vectors
produces the vector sum, and multiplying two vectors produces the dot
product
\begin{displaymath}
(x_1,y_1)\cdot (x_2,y_2) \mapsto x_1x_2+y_1y_2
\end{displaymath}
Multiplying
a number times a vector, or a vector times a number, should scale the
vector by the number.  Adding a vector plus a number is not defined.
Also define a generic function length, such that the length of a
vector is its length and the length of a number is its absolute value.


\section{2. The TOOL Interpreter}

A complete listing of the TOOL interpreter is appended to this problem
set.  This section leads you through the most important parts,
describing how they differ from the Scheme evaluator in chapter 4.

\subsubsection{EVAL and APPLY}

We've named the eval procedure {\tt tool-eval} so as not to confuse it
with Scheme's ordinary {\tt eval}.  The only difference between {\tt
tool-eval} and the {\tt eval} in chapter 4 are the new cases added to
handle the new special forms: {\tt define-generic-function}, {\tt
define-method}, {\tt define-class}, and {\tt make}.  Each clause dispatches
to the appropriate handler for that form.  Note that we have deleted
{\tt lambda}; all TOOL functions are defined with {\tt
define-generic-function}.\footnote{Omitting {\tt lambda} takes away
our ability to have unnamed procedures, as we do in Scheme.  You might
want to think about how to add such a feature to TOOL.}

\beginlisp
(define (tool-eval exp env)
  (cond ((self-evaluating? exp) exp)
        ((quoted? exp) (text-of-quotation exp))
        ((variable? exp) (lookup-variable-value exp env))
        ((definition? exp) (eval-definition exp env))
        ((assignment? exp) (eval-assignment exp env))
        ;;((lambda? exp) (make-procedure exp env))      ;We don't need lambda!
        ((conditional? exp) (eval-cond (clauses exp) env))
        ((generic-function-definition? exp)               ;DEFINE-GENERIC-FUNCTION
         (eval-generic-function-definition exp env)) 
        ((method-definition? exp) (eval-define-method exp env)) ;DEFINE-METHOD
        ((class-definition? exp) (eval-define-class exp env))     ;DEFINE-CLASS
        ((instance-creation? exp) (eval-make exp env))          ;MAKE
        ((application? exp)
         (tool-apply (tool-eval (operator exp) env)
                     (map (lambda (operand) (tool-eval operand env))
                          (operands exp))))
        (else (error "Unknown expression type -- EVAL >> " exp))))
\endlisp

{\tt Apply} also gets an extra clause that dispatches to a
procedure that handles applications of generic functions.


\beginlisp
(define (tool-apply procedure arguments)
  (cond ((primitive-procedure? procedure)
         (apply-primitive-procedure procedure arguments))
        ((compound-procedure? procedure)
         (eval-sequence
          (procedure-body procedure)
          (extend-environment (parameters procedure)
                              arguments
                              (procedure-environment procedure))))
        ((generic-function? procedure)
         (apply-generic-function procedure arguments))
        (else (error "Unknown procedure type -- APPLY"))))
\endlisp

\subsubsection{New data structures}

A {\it class} is represented by a data structure that contains the
class name, a list of slots for that class, and a list of all the
ancestors of the class.  For instance, in our cat example above, we
would have a class with the name {\tt <house-cat>}, slots {\tt
(address size breed)}, and superclasses {\tt (<cat> <object>)}.  Note
that the slot names include {\it all} the slots for that class (i.e.,
including the slots for the superclass).  Similarly, the list of
ancestors of a class includes the superclass and all of its ancestors.

A {\it generic function} is a data structure that contains the name of
the function and a list of the methods defined for that function.
Each method is a pair---the specializers and the resulting procedure
to use.  The specializers are a list of classes to which the arguments
must belong in order for the method to be applicable.  The procedure
is represented as an ordinary Scheme procedure.

An instance is a structure that contains the class of the instance and
the list of values for the slots.

See the attached code for details of the selectors and constructors
for these data structures.

\subsection{Defining generic functions and methods}

The special form:

\beginlisp
(define-generic-function {\it name})
\endlisp

\noindent
is handled by the following procedure:

\beginlisp
(define (eval-generic-function-definition exp env)
  (let ((name (generic-function-definition-name exp)))
    (let ((val (make-generic-function name)))
      (define-variable! name val env)
      (list 'defined 'generic 'function: name))))
\endlisp

This procedure extracts the {\it name} portion of the expression and
calls {\tt make-generic-function} to create a new generic function.
Then it binds {\it name} to the new generic function in the given
environment.  The value returned is a message to the user, which will
be printed by the read-eval-print loop.

{\tt Eval-define-method} handles the special form

\beginlisp
(define-method {\it generic-function} ({\it params-and-classes}) . {\it body})
\endlisp

for example

\beginlisp
(define-method say ((cat <cat>) (stuff <number>))
          (print '(cats never discuss numbers)))
\endlisp

In general here, {\it generic-function} is the generic function to
which the method will be added, {\it params-and-classes} is a list of
parameters for this method and the classes to which they must belong,
and {\it body} is a procedure body, just as for an ordinary Scheme
procedure.\footnote{The dot before the word ``body'' signifies that we
can put more than one expression in the body---just as with ordinary
Scheme procedures.} The syntax procedures for this form include
appropriate procedures to select out these pieces (see the code).

{\tt Eval-define-method} first finds the generic function.  Notice
that the {\it generic-function} piece of the expression must be
evaluated to obtain the actual generic function.  {\tt
Eval-define-method} disassembles the list of {\it params-and-classes}
into separate lists of parameters and classes.  The parameters, the
{\it body}, and the environment are combined to form a procedure, just
as in Scheme.  The classes become the specializers for this method.
Finally, the method is installed into the generic function.

\beginlisp
(define (eval-define-method exp env)
  (let ((gf (tool-eval (method-definition-generic-function exp) env)))
    (if (not (generic-function? gf))
        (error "Unrecognized generic function -- DEFINE-METHOD >> "
               (method-definition-generic-function exp))
        (let ((params (method-definition-parameters exp)))
          (install-method-in-generic-function
           gf
           (map (lambda (p) (paramlist-element-class p env))
                params)
           (make-procedure (make-lambda-expression
                            ;;extract the parameter names from the paramlist
                            (map paramlist-element-name params)
                            (method-definition-body exp))
                           env))
          (list 'added 'method 'to 'generic 'function:
                (generic-function-name gf))))))
\endlisp

\paragraph{Tutorial exercise 3:} {\tt Eval-define-method} calls {\tt
paramlist-element-class} in order to find the class for each
parameter.  Without looking at the attached code, predict whether
{\tt paramlist-element-class} should call {\tt tool-eval}.  Now look
at the code and see if you were right.  Give a careful explanation of
why {\tt tool-eval} is (or is not) called, and what difference this
makes.

\subsection{Defining classes and instances}

The special form

\beginlisp
(define-class {\it name} {\it superclass} . {\it slots})
\endlisp

\noindent is handled by

\beginlisp
(define (eval-define-class exp env)
  (let ((superclass (tool-eval (class-definition-superclass exp)
                               env)))
    (if (not (class? superclass))
        (error "Unrecognized superclass -- MAKE-CLASS >> "
               (class-definition-superclass exp))
        (let ((name (class-definition-name exp))
              (all-slots (collect-slots
                          (class-definition-slot-names exp)
                          superclass)))
          (let ((new-class
                 (make-class name superclass all-slots)))
            (define-variable! name new-class env)
            (list 'defined 'class: name))))))
\endlisp

The only tricky part here is that we have to collect all the slots from
all the ancestor classes to combine with the slots declared for this
particular class.  This is accomplished by the procedure {\tt
collect-slots} (see the code).

The final special form

\beginlisp
(make {\it class} {\it slot-names-and-values})
\endlisp

\noindent is handled by the procedure {\tt eval-make}.  This
constructs an instance for the specified class, with the designated slot
values.  See the attached code for details.

\smallskip
\begin{center}
{\bf REST STOP}
\end{center}
\smallskip

\subsection{Applying generic functions}

Here is where the fun starts, and what all the preceding machinery was
for.  When we apply a generic function to some arguments, we first
find all the methods that are applicable, given the classes of the
arguments.  This gives us a list of methods, of which we will use the
first one.  (We'll see why the first one in a minute.)  We extract the
procedure for that method and apply that procedure to the arguments.
Note the subtle recursion here: {\tt apply-generic-function} (below)
calls {\tt tool-apply} with the procedure part of the method.

\beginlisp
(define (apply-generic-function generic-function arguments)
  (let ((methods (compute-applicable-methods-using-classes
                  generic-function
                  (map class-of arguments)))) 
    (if (null? methods)
        (error "No method found -- APPLY-GENERIC-FUNCTION")
        (tool-apply (method-procedure (car methods)) arguments))))
\endlisp


To compute the list of ``applicable methods'' we first find all
methods for that generic function that can be applied, given the list
of classes for the arguments.  We then sort these according to an
ordering called {\tt method-more-specific}.  The idea is that the
first method in the sorted list will be the most specific one, which
is the the best method to apply for those arguments.

\beginlisp
(define (compute-applicable-methods-using-classes generic-function classes)
  (sort
   (filter
    (lambda (method)
      (method-applies-to-classes? method classes))
    (generic-function-methods generic-function))
   method-more-specific?))
\endlisp

To test if a method is applicable, given a list of classes of the
supplied arguments, we examine the method specializers and see
whether, for each supplied argument, the class of the argument is a
subclass of the class required by the specializer:

\beginlisp
(define (method-applies-to-classes? method classes)
  (define (check-classes supplied required)
    (cond ((and (null? supplied) (null? required)) true) ;all chalked
          ((or (null? supplied) (null? required)) false) ;something left over
          ((subclass? (car supplied) (car required))
           (check-classes (cdr supplied) (cdr required)))
          (else false)
          ))
  (check-classes classes (method-specializers method)))
\endlisp

To determine subclasses, we use the class ancestor list: {\tt class1}
is a subclass of {\tt class2} if {\tt class2} is a member of the class
ancestor list of {\tt class1}:

\beginlisp
(define (subclass? class1 class2)
  (or (eq? class1 class2)
      (memq class2 (class-ancestors class1))))
\endlisp

Finally, we need a way to compare two methods to see which one is
``more specific.''  We do this by looking at the method specializers.
{\tt Method1} is considered to be more specific than {\tt method2} if,
for each class in the list of specializers, the class for {\tt
method1} is a subclass of the class for {\tt method2}.  (See the
procedure {\tt method-more-specific?} in the attached code.)

\paragraph{Tutorial exercise 4:} In the example at the end of
section 1, explain how the generic function dispatch chooses the
correct {\tt say} method when we ask the cat {\tt fluffy} to say a
number.  In particular, what are all the applicable methods?  In what
order will they appear after they are sorted according to {\tt
method-more-specific}?


\subsection{Classes for Scheme data}

TOOL is arranged so that ordinary Scheme data objects---numbers,
symbols, and so on---appear as TOOL objects.  For example, any number
is an instance of a predefined class called {\tt <number>}, which is
a class with no slots, whose superclass is {\tt <object>}.  The TOOL
interpreter accomplishes this by having a special set of classes,
called {\tt scheme-object-classes}.  If a TOOL object is not an
ordinary instance (i.e., an instance data structure as described
above), the interpreter checks whether it belongs to one of the
Scheme object classes by applying an appropriate test.  For example,
anything that satisfies the predicate {\tt number?} is considered to
be an instance of {\tt <number>}.  See the code for details.

\subsection{Initial environment and driver loop}

When the interpreter is initialized, it builds a global environment
that has bindings for {\tt true}, {\tt false}, {\tt nil}, the
pre-defined classes, and the initial set of generic functions listed
at the end of section 1.  The driver loop is essentially the same as
the {\tt driver-loop} procedure in chapter 4 of the notes.  One cute
difference is that this driver loop prints values using the TOOL
generic function {\tt print}.  By defining new methods for {\tt
print}, you can change the way the interpreter prints data objects.

\paragraph{Tutorial exercise 5:} Define a {\tt print} method so that
TOOL will print vectors (which you defined in exercise 2) showing
their xcor and ycor.

\section{3. To do in lab}

When you load the code for this problem set, the entire TOOL
interpreter code (attached) will be loaded into Scheme.  However, in
order to do the lab exercises, you will need to modify only a tiny bit
of it.  This code has been separated out in the file {\tt mod.scm},
so you can edit it conveniently.

To start the TOOL interpreter, type {\tt (initialize-tool)}.  This
initializes the global environment and starts the read-eval-print
loop.  To evaluate a TOOL expression, type it after the prompt,
followed by {\sc ctrl-}{\tt x} {\sc ctrl-}{\tt e}.

In order to keep the TOOL interpreter simple, we have not provided any
mechanism for handling errors.  Any error (such as an unbound
variable) will bounce you back into Scheme's error handler.  To get
back to TOOL, quit out of the error and restart the driver loop by
typing {\tt (driver-loop)}.  If you make an error that requires
reinitializing the environment, you can rerun {\tt initilialize-tool},
but this will make you lose any new classes, generic functions, or
methods you have defined.

\paragraph{Lab exercise 6:}  Start the TOOL evaluator and try out
your vector definitions from exercise 2.  Also, check your answer to
exercise 5, where you defined a new print method for vectors.  How did
TOOL print vectors before you added your own print method?  Turn in
your definitions and a brief interaction showing that they work.

\paragraph{Lab exercise 7:}  One annoying thing about TOOL is that if
you define a method before you've defined a generic function for that
method, you will get an error.  For example, in the first example
in section 1, we had to explicitly evaluate

\beginlisp
(define-generic-function 4-legged?)
\endlisp

\noindent before we could evaluate

\beginlisp
(define-method 4-legged? ((thing <object>))
  'Who-knows?)
\endlisp

\noindent Otherwise, the second expression would give the error that
{\tt 4-legged?} is undefined.  Modify the TOOL interpreter so that,
if the user attempts to define a method for a generic function that
does not yet exist, TOOL will first automatically define the generic
function.  One thing to consider: In which environment should the
name of the generic function be bound: the global environment, the
environment of the evaluation?  some other environment?  There is no
``right answer'' to this question---{\it you} are the language
designer.  But whatever choice you make, write a brief paragraph
justifying your choice.  In particular, include an example of a
program for which the choice of environment matters, i.e., where the
program would have a different behavior (or perhaps give an error) if
the choice were different.  (Hint: The only procedure you should need
to modify for this exercise is {\tt eval-define-method}.)  Turn in,
along with your design justification, your modified code together
with a brief interaction showing that the modified interpreter works
as intended.)

\paragraph{Lab exercise 8:} Another inconvenience in TOOL is that we
need to use {\tt get-slot} in order to obtain slot values.  It would
be more convenient to have TOOL automatically define selectors for
slots.  For example, it would be nice to be able to get the x and y
coordinates of a vector by typing {\tt (xcor v)} and {\tt (ycor v)}
rather than {\tt (get-slot v 'xcor)} and {\tt (get-slot v 'ycor)}.
Modify the interpreter to do this.  Namely, whenever a class is
defined, TOOL should automatically define a generic function for each
of its slot names, together with a method that returns the
corresponding slot value for arguments of that class.  Turn in a
listing of your code and an example showing that it works.  (Hint:
The only part of interpreter you need to modify for this exercise is
{\tt eval-define-class}.)

\paragraph{Lab exercise 9:} Give some simple example of defining some
objects and methods (besides cats and vectors) that involve
subclasses, superclasses, and methods, and which illustrate the
modifications you made in exercises 8 and 9.

\section{4. Multiple Superclasses: To do AFTER you are done in the lab}

\fbox{
This final question asks you to consider a tricky issue in language
design.  We are not requiring you to actually implement your design.
Nevertheless, we do expect you to think carefully about the issues
involved and to give a careful description of the solution you come up
with.  Don't think that this is a straightforward exercise---designers
of object-oriented languages are still arguing about it.  }

\smallskip

The major way in which TOOL is simpler than other object-oriented
languages such as Dylan or the Common Lisp Object System (CLOS) is
that each class has only {\it one} immediate superclass.  As
illustrated with message-passing systems (lecture on October 22),
there are cases where it is convenient to have a class inherit
behavior from more than one kind of class.

This will involve some changes to TOOL.  As a start, the syntax for
{\tt define-class} must be modified to accept a list of superclasses
rather than a single superclass.  Let's assume that {\tt define-class}
now takes a list of superclasses.  For instance, going back to our
original example about cats, we might have:

\beginlisp
(define-class <fancy-house-cat> (<house-cat> <show-cat>))
\endlisp

\noindent
This new class should inherit from both {\tt <house-cat>} and {\tt
<show-cat>}.  In general, when the new class is constructed, it should
inherit methods and slots from {\it all} its superclasses (and their
ancestors).

However, it's not obvious what inheritance should mean.  For example,
suppose we have a generic function {\tt eat} and we define methods as
follows:

\beginlisp
(define-method eat ((c <house-cat>))
   (print '(yum: I'm hungry)))
\null
(define-method eat ((c <show-cat>))
   (print '(I eat only caviar)))
\endlisp

\noindent What should happen when we ask a fancy-house-cat (which is
both a show-cat and a house-cat) to eat?  More generally, what is the
``most specific method'' that should be used when a generic function
is applied to its arguments, given that some of the arguments may have
multiple superclasses?  What are the new kinds of choices that arise?
How should the language give the user the ability to control these
choices?  (Or maybe it {\it shouldn't} give the user this level of
control.)

\paragraph{Post-lab exercise 10} You are now a language designer.
Your task is to design an extension to TOOL so that it handles
classes with multiple superclasses.  Three of the issues you have to
deal with are: (a) What should be the syntax for defining classes?
(b) What slots does a class get when it is defined?  (c) How is a
method chosen when a generic function is applied to its arguments?
Prepare a design writeup that has three parts:

\begin{enumerate}

\item Write a clear 2--3 page description of your language extension.
This description should be geared toward the {\it user} of the
language.  It should include a simple, but realistic and non-trivial
example of a program that involves multiple superclasses.  The
example should illustrate how your language handles each of the
three issues (a), (b), and (c).  You should also explain how the
language deals with each of these issues in general.

\item For each of design choices you illustrated in part 1, give an
{\it alternative} choice you could have made, and explain briefly why
you think your choice is better.  If you can't think of any other
choice you might have made, then say so.

\item As carefully as you can (but without actually writing any code)
specify the procedure that the evaluator should follow in choosing
which method to select when applying a generic function to a given
set of arguments.  Your description should be clear enough so that
someone could implement this procedure based upon your specification.

\end{enumerate}

\paragraph{Optional extra credit}  Implement your design for multiple
superclasses in TOOL and demonstrate that it works.  The TOOL
interpreter was designed to make this not too difficult, but it will
involve a considerable number of small changes to the code and is
likely to be time-consuming.


\end{document}





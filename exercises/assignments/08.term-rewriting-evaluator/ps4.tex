% Document Type: LaTeX
% Master File: ps4a.tex
\input ../6001mac
\usepackage{../epsfig}
\newcommand{\Code}[1]{{\tt #1}}

\outer\def\beginlispbig{%
  \begin{minipage}[t]{\linewidth}
  \begin{list}{$\bullet$}{%
    \setlength{\topsep}{0in}
    \setlength{\partopsep}{0in}
    \setlength{\itemsep}{0in}
    \setlength{\parsep}{0in}
    \setlength{\leftmargin}{1.5em}
    \setlength{\rightmargin}{0in}
    \setlength{\itemindent}{0in}
  }\item[]
  \obeyspaces
  \obeylines \tt}


\def\fbox#1{%
  \vtop{\vbox{\hrule%
     \hbox{\vrule\kern3pt%
 \vtop{\vbox{\kern3pt#1}\kern3pt}%
 \kern3pt\vrule}}%
 \hrule}}


\begin{document}

\psetheader{Sample Problem Set}{Rewriting Problem Set}
\begin{center}\large
{\bf Rewriting Symbolic Expressions}
\end{center}

\medskip

\noindent

This problem set is about handling symbolic expressions as abstract
data.  The main application will be to the Term Rewriting Model (TRM)
for a language similar to Scheme, as described in the Supplementary
Notes handout (which you should read before reading further in this
Problem Set).  In this problem set, the abstract data represents the
subset of Scheme expressions specified in the handout.

Loading the problem set will allow you to run an implementation of the
TRM.  There are only two parts of the implementation you need to learn
about for this problem set; these are in the attached files
\Code{smeval.scm} and \Code{smsyntax.scm}.

The TRM can be used to observe selected steps in the execution of a
{\em body}.  (A body is a sequence of zero or more definitions followed by
an expression.)  For example, we can use the procedure {\tt smeval} to
observe the execution of the factorial program as follows:

\beginlispbig
(smeval
 '(
   (define (fact n)
     (if (= n 0)
         1
         (* n (fact (- n 1)))))
   (fact 3)
   ))
\endlisp

The resulting printout shows selected intermediate steps in the
execution, and the final value

\beginlispbig
;==(1)==>
((define fact
   (lambda (n)
     (if (= n 0)
         1
         (* n (fact (- n 1))))))
 ((lambda (n)
    (if (= n 0) 
        1
        (* n (fact (- n 1))))) 3))
\null
;==(2)==>
((define fact
   (lambda (n)
     (if (= n 0)
         1
         (* n (fact (- n 1))))))
 (define n\#8 3)
 (if (= n\#8 0)
     1
     (* n\#8 (fact (- n\#8 1)))))
\null
... Lots of steps ...
\null
;==(30)==>
((define fact
   (lambda (n)
     (if (= n 0)
         1
         (* n (fact (- n 1))))))
 (define n\#8 3)
 (define n\#9 2)
 (define n\#10 1)
 (define n\#11 0)
 (* 3 (* 2 1)))
\null
;==(33)==>
6
Syntactic Value was returned
;Value: 6
\endlisp

The procedure \Code{smeval} constructs a list of selected ``steps''
where each step consists of a step number $n$ and the body which
results after $n$ reduction steps, then prints the steps, and returns
the final body.  The main procedure of the TRM implementation is
\Code{one-step-body}.  When applied to a body and a list of
definitions, it does a single rewrite step, if possible.  For example:

\beginlispbig
(pp 
 (one-step-body
  '(
    (define (fact n)
      (if (= n 0)
          1
          (* n (fact (- n 1)))))
    ((lambda (n)
       (if (= n 0)
           1
           (* n (fact (- n 1)))))
     3)
    )
  '()))
(stepped
 ((define (fact n)
    (if (= n 0)
        1
        (* n (fact (- n 1)))))
  (define n\#13 3)
  (if (= n\#13 0)
      1
      (* n\#13 (fact (- n\#13 1))))))
;Value: \#[useless-value]
\endlisp

The value returned by \Code{one-step-body} is a ``tagged body'', i.e.,
a list of a ``tag'' and either a body (as defined above) or a data
structure describing an error condition.  The tag is one of the three
symbols \Code{stepped}, \Code{val}, or \Code{stuck}.  If the tag is
either \Code{stepped} or \Code{val} the second part of the pair is a
body.  If the tag is \Code{stuck} then the second part is an error
description.  If the tag is \Code{val} then the second part of the
pair is a ``fully simplified'' body, i.e., no further computational
steps can be done.

\subsubsection{Abstract Syntax}

The Scheme printer prints out lists using the kind of
matched-parenthesis syntax that the Scheme reader expects as input for
evaluation.  So when we want to process expressions themselves as
syntactic data---as opposed to evaluating them---it is natural to
represent them as list-structures which ``look the same.''  For example,
we represent the lambda expression \Code{(lambda (n) (+ n n))} by
the list-structure whose box-and-pointer diagram is:

\begin{figure}[hbtp]
\center{\epsfig{file=cons.ps,width=6in}}
\end{figure}

It is sometimes useful to allow for additional representations of
expressions.  One can treat expressions as abstract data objects which
are constructed and accessed using abstract constructors and
selectors.  Some examples of abstract constructors and selectors are
given below.  First, we define the procedure \Code{tagged-pair?}:

\beginlispbig
(define (tagged-pair? tag)
  (lambda (exp)
    (and (pair? exp)
         (eq? (car exp) tag))))
\endlisp

\noindent
We can use this to easily make a test for lambda expressions.

\beginlispbig
(define lambda-expression? (tagged-pair? 'lambda))
\endlisp

\noindent
or we can define a selector for the formals of lambda expression:

\beginlispbig
(define formals-of-lambda cadr)
\endlisp

\noindent
Similarly, we can test whether an expression is a conditional:

\beginlispbig
(define if? (tagged-pair? 'if))
\endlisp

\noindent
and we can define a constructor and selectors for conditionals:

\beginlispbig
(define (make-if test consequent alternative)
  (list 'if test consequent alternative))
\null
(define test-of-if cadr)
(define consequent-of caddr)
(define alternative-of cadddr)
\endlisp

Scheme code written using standard test/selector/constructor names
like those defined above leads to more understandable and {\em
debuggable} code than if we tried to save characters by typing, say,
\Code{cadddr} instead of \Code{alternative-of}.  All of the code for
the TRM implementation (with one exception which you will be asked to
rectify) abides by the discipline that expressions be treated as {\em
abstract data}, to be examined and manipulated only by
test/selector/constructor procedures which are given as part of the
abstract data specification.

\section{The Assignment}

For each of the exercises below in which you write Scheme programs, turn
in a listing of your programs along with test results demonstrating that
they work.

\paragraph{Exercise 1:} From the 6.001 Edwin editor, use {\tt M-x
load-problem-set} to load the code for Problem Set 4.  You should now
have a file \Code{tests.scm} in which you will find a recursive and an
iterative definition of factorial.  The procedure \Code{smeval}
constructs a list of selected steps.  Whether a given step is selected
for inclusion is determined by the procedure \Code{save-this-step?}
which takes two arguments, a step number and a body.  Modify the
procedure
\Code{save-this-step?} so that every step is saved and then try
\Code{smeval}ing factorial of 4 using each of the two definitions
given in \Code{tests.scm}.

\paragraph{Exercise 2:} You might note that the examples on
pages 32 and 33 of the text only show a few interesting steps in a
computation of factorial.  However, the steps selected for printing by
\Code{smeval} are too many and show too much detail to convey the insights.  
In this problem we will develop a version of \Code{save-this-step?}
and a version of the printer that that selects only interesting steps
and prints them in a way that develops figures similar to Fig. 1.3 and
Fig. 1.4. of the text.

\paragraph{2.A} Here we write a predicate procedure
\Code{simple?} which tells us whether a given expression is
simple.  By ``simple'' we will mean that the expression is composed
entirely of numbers, the truth values \Code{\#t} and \Code{\#f},
variables, and combinations.  The file \Code{smsyntax.scm} contains
abstract constructors, selectors, and predicates such
\Code{variable?}, \Code{combination?}, \Code{operator}, and
\Code{operands} that we may use in the definition of \Code{simple?}.
(Note that \Code{number?} and \Code{boolean?} are primitives in
Scheme.)

We will use the \Code{simple?} predicate in a new definition of
\Code{save-this-step?} as follows: 

\beginlispbig
(define (save-this-step? step-number body)
  (simple? (expression-of-body body)))
\endlisp

Give an appropriate definition of \Code{simple?}.  Install it and the
definition of \Code{save-this-step?} that uses it (above). 

After you make this change show an execution of

\beginlispbig
(smeval 
  '((define fact 
      (lambda (n) 
        (if (= n 0)
            1
            (* n (fact (- n 1))))))
    (fact 4)))
\endlisp

\paragraph{2.B} The bodies being printed contain the definitions, but
we only want to see the expressions to illustrate our point.  Pick up
the definition of \Code{print-stepped-message} from the file
\Code{smeval.scm} and modify it to only print the final expression
part of the body.  Demonstrate your change on the recursive factorial
example you used to demonstrate modifications for part 2.A.  Also show
an iterative factorial computation, a recursive
fibonacci computation (be careful---do not try to do more than 
\Code{(fib 4)}---we don't want to kill many trees!), and an iterative
fibonacci computation.


\paragraph{2.C} Unfortunately, the specification of \Code{simple?}
given in part 2.A admits expressions that contain ugly generated
variables, such as

\beginlispbig
(* 4 (* 3 (* 2 (* n\#34 (fact (- n\#34 1)))))).
\endlisp

Write a version of \Code{simple?} that excludes these cases, by
excluding any expression with a variable in an operand position.
Demonstrate your change on the same examples you used to demonstrate
modifications for part 2.B.  Notice that the recursive fibonacci
computation does not show the expected complexity (although it is
really there!).  Explain why the recursive computation looks simpler
than it actually is with this version of \Code{simple?}.

\paragraph{Exercise 3.A:} The \Code{desugar} procedure in
\Code{smsyntax.scm} desugars \Code{let}'s into lambda applications,
and \Code{cond}'s into \Code{if}'s.  The Revised$^4$ Report on the
Algorithmic Language Scheme explains how
several further Scheme constructs can be desugared into the subset of
Scheme handled by the TRM (which now includes \Code{let}'s and
\Code{cond}'s, since we already know how to desugar them).  Pick one
of these further constructs, and add a case to \Code{desugar} to
handle it.  Note that \Code{desugar} treats bodies as abstract data
which it manipulates only by standard constructors and selectors; your
revision of it should also conform to this discipline.

A straightforward form to desugar is \Code{let*}.  Some of the other
desugarings, such as the one for \Code{and}, are a little tricky
because they are designed to work in the general case with
side-effects.  For our TRM without side-effects, a satisfactory
desugaring of \Code{(and $A$ $B$ \ldots)} is \Code{(let ((x $A$)) (if
  x (and $B$ \ldots) x)}, where the inner \Code{and} must itself be
recursively desugared.

\paragraph{Exercise 3.B:} Test your definition of \Code{desugar}.
(Some tests for \Code{let*} are included in \Code{tests.scm}.)

\end{document}
















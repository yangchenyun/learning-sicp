\documentclass[11pt]{article}
\usepackage{6001}
\usepackage{time,fullpage}
\usepackage{amsmath}
\usepackage{../epsfig}
%\input ../6001mac

\textheight 8in

\oddsidemargin 11pt
\evensidemargin \oddsidemargin
\marginparwidth 0.5in

\advance\textwidth by -2\oddsidemargin

\newdimen\tablewidth
\tablewidth=\textwidth
\advance \tablewidth by -11pt

\parindent 0.0in
\parskip \medskipamount

\let\to\rightarrow
\let\union\cup
\let\cross\times

\def\SI{\mbox{Scheme-Integer}}
\def\SNI{\mbox{Scheme-Nonneg-Integer}}
\def\SN{\mbox{Sch-\-Num}}
\def\SB{\mbox{Sch-Bool}}
\def\Empty{\mbox{Empty}}
\def\SI{\mbox{Sch-Int}}
\def\SNI{\mbox{Sch-Nonneg-Int}}


\makeatletter

\def\DefineBNFAbbrev#1#2{%
    \expandafter\newcommand\expandafter{\csname#1\endcsname}{\@bnfabbrev{#2}}%
}

\def\@bnfabbrev#1{%
    \@ifnextchar _{\@subabbrev{#1}}{\@plainabbrev{#1}}%
}

\def\@subabbrev#1_#2{\hbox{\normalfont$\langle\mbox{#1}_{#2}\rangle$}}

\def\@plainabbrev#1{\hbox{\normalfont$\langle\mbox{#1}\rangle$}}

% \def\exp{\hbox{$\bnf <expression>$}}

\DefineBNFAbbrev{expr}{expression}
\DefineBNFAbbrev{define}{define}
\DefineBNFAbbrev{defines}{defines}
\DefineBNFAbbrev{vrbl}{variable}
\DefineBNFAbbrev{bod}{body}
\DefineBNFAbbrev{init}{init}
\DefineBNFAbbrev{val}{value}
\DefineBNFAbbrev{pval}{primitive\ value}
\DefineBNFAbbrev{ival}{immediate\ value}
\DefineBNFAbbrev{test}{test}
\DefineBNFAbbrev{orsf}{or}
\DefineBNFAbbrev{andsf}{and}
%\DefineBNFAbbrev{beginsf}{begin}

\makeatletter


\def\fbox#1{%
  \vtop{\vbox{\hrule%
     \hbox{\vrule\kern3pt%
 \vtop{\vbox{\kern3pt#1}\kern3pt}%
 \kern3pt\vrule}}%
 \hrule}}


\begin{document}

\psetheader{Spring Semester, 1995}{Supplementary Notes}
\begin{center}\large
{\bf A Term Rewriting Model for Scheme}
\end{center}

\medskip


\begin{flushleft}
Issued:  Tuesday, 7 March, 1995\\
\end{flushleft}

\section{Substitution Models}

A useful way to think about computation in Scheme is as {\em algebraic
  simplification\/} aimed at computing the value of an expression by
successively simplifying its subexpressions.  The use of names for terms
is also a familiar part of arithmetic/algebraic manipulation, as in,
\begin{quote}
Let $p(x) = x^{2}-1$, then the polynomial $p$ factors into
$p_{1}\cdot p_{2}$ where $p_{1}(x) = x + 1$ and $p_{2}(x) = x - 1$.
\end{quote}
Naming rules are usually not spelled out in studying algebra, but in
Scheme, as in all programming languages, such rules play a key
role---arguably a more important one than the ordinary algebraic rules.  A
system of rules which explain Scheme computation in terms of symbolic
manipulation of Scheme expressions and definitions is called a {\em
Substitution Model} for Scheme.

For example, the {\em variable instantiation} rule is a basic naming rule:
\begin{quote}
replace an occurrence of a variable by the expression it names.
\end{quote}
Implicit in this rule is a simple recipe for figuring out what expression
a variable names: starting at the variable to be instantiated, search back
through the expression tree until you come to a definition of that
variable.

In the Substitution Model sketched in 6.001 lectures and text, there is
also a particular {\em substitution rule} for handling procedure
application.  In the case of a one argument procedure, the substitution
rule is:
\begin{quote}
rewrite ``\Code{((lambda ($x$) $B$) $V$)}'' as ``$B$ with $V$ substituted
for $x$''.
\end{quote}

In these notes we describe a slightly different rule for procedure
application.  The Substitution Model using this new rule dramatically
shortens the size of expressions to be manipulated in many examples.
It also will allow us to model Scheme's computational behavior much
more accurately\footnote{The Substitution Model in the text is
  sketchily described (on purpose, to avoid distracting beginners with
  details).  The most straightforward ways of spelling out the details
  lead to models which have to be abandoned altogether when we
  introduce side-effects.  Side-effects are explained in Chapter 3 of
  the text with a new {\em environment model}.  You won't have to
  unlearn the new Substitution Model described in these Supplementary
  Notes, because it extends to handle side-effects quite well.  In
  fact, the environment model can usefully be understood as a way to
  {\em implement} the new Substitution Model efficiently.}.

We'll simply call the new rule the {\em lambda application rule}.  In the
one argument case, it is:
\begin{quote}
rewrite ``\Code{((lambda ($x$) $B$) $V$)}'' as ``$B$''\\
while adding the definition ``\Code{(define $x$ $V$)}.''
\end{quote}
Of course to use this rule, we'll need to specify where and how to ``add''
definitions.

It's helpful to realize that the new lambda application rule could be used
to simulate the old rule: starting with \Code{((lambda ($x$) $B$) $V$)},
use the lambda application rule.  This gives $B$ along with the definition
\Code{(define $x$ $V$)}.  Then, by successively instantiating all the
$x$'s in $B$ by copies of $V$, we arrive at the same result as the
substitution rule.  Well almost---we still have the definition of $x$
hanging around.  But since we instantiated all the $x$'s, this definition
no longer refers to anything and is not needed.  So we will apply a new
rule which allows us to remove unreferenced definitions.

\subsection{Syntactic Values}

We said that the point of Substitution Model rewrite rules is to arrive at
the value of an expression.  More precisely, the aim is to arrive at some
standardized, syntactic representation called the {\em syntactic value} of
the expression.  Before examining the rules in detail, we should have a
clear picture of what syntactic values look like.

The most familiar syntactic values are the {\em primitive values}.  These
are objects like numbers or truth values which have standard
representations printable as output.

Scheme provides dozens of {\em primitive procedure variables} like
\Code{+}, \Code{<}, \Code{atan}, or \Code{not} which can be applied to
primitive values.  These primitive procedure variables will also be
syntactic values\footnote{We assume here that primitive procedure
variables won't be redefined!  In fact, according to the official
definition of \RFS, it is possible to redefine them.  In creating 6.001
Scheme from general MIT Scheme, the arithmetic operators do get redefined
slightly---so that rational arithmetic does not yield unexpectedly huge
quotients.}.

Scheme also provides a means to construct procedure values, namely, with
lambda expressions.  The
\begin{itemize}
\item primitive values,
\item primitive procedure variables, and
\item lambda expressions,
\end{itemize} are called {\em immediate values}.

The full meaning of lambda expression like \Code{(lambda (x) (+ x y))}
isn't determined until we know what number \Code{y} is defined to be, so
we also allow syntactic values like
\begin{lisp}
(define y 2)
(lambda (x) (+ x y))
\end{lisp}
which represents the ``add 2'' function.  In general, a {\em syntactic
value} is a sequence of zero or more {\em immediate definitions} followed
by an immediate value, where an immediate definition is of the form
\Code{(define \vrbl\ \ival)}.


\section{Term Rewriting}

We begin by describing rewrite rules for combinations and \Code{if}
expressions.  To manage definitions, we'll also have rewrite rules for
Scheme bodies, where a {\em body} is sequence of zero or more definitions
followed by an expression.  We'll use the notation
\[\bod_{1} \rulesto \bod_{2}\]
to indicate the rule that $\bod_{1}$ can be rewritten as, that is, replaced
by, $\bod_2$.

When an expression not only returns a value but also causes {\em
side-effects} like printing, then it is essential to be able to control
how rules are applied, and this is a crucial aspect of Scheme programming.
But as long as we focus on {\em values}, as we have to this point in
6.001, we needn't specify any special strategy for applying rules: a
sub-body matching the lefthand side of a rule can be rewritten as the
righthand side, anywhere, anytime.\footnote{For example, Scheme
unambiguously specifies that evaluation of the combination:
\[\Code{((lambda (x) ((lambda (y) 99) (display \#t))) (display \#f))}\]
will cause the operand to be evaluated before the {\em body} of the
operator, so \Code{\#f\#t} will be printed before the value \Code{99} is
returned.  On the other hand, if the inner \Code{display} combination
within the body was evaluated first, then \Code{\#t\#f} would be printed
before the value \Code{99} was returned.  But it's the same syntactic
value, \Code{99}, returned in either case!} A Substitution Model in which
rules can be used anywhere is called a {\em term rewriting model} (TRM).
That's what we will describe for Scheme.

\subsection{Primitive Application Rules}
We take for granted an (infinite) set of {\em primitive application rules\/}
for rewriting primitive operators on primitive values, for example,
\begin{eqnarray*}
\Code{(- 2 3)}  & \rulesto & \Code{-1}\\
\Code{(* 4 5 42)}  & \rulesto & \Code{840}\\
\vdots
\end{eqnarray*}
In general, a primitive application rule is of the form
\[\bnf\Code{($<primitive procedure variable>$ \pval_{1} \ldots)} \rulesto \pval.\]

By repeatedly {\em applying\/} such rules, namely replacing lefthand sides
of rules with their righthand sides, we can rewrite any nested primitive
combination to its primitive value.  For example,
\begin{eqnarray*}
\Code{(+ 0 1 (- 2 3) (* 4 5 (* -6 -7)))}  & \rulesto & \Code{(+ 0 1 (- 2 3) (* 4 5 42))}\\
{}&  \rulesto & \Code{(+ 0 1 -1 (* 4 5 42))}\\
{}&  \rulesto & \Code{(+ 0 1 -1 840)}\\
{}&  \rulesto & \Code{840}
\end{eqnarray*}

Another familiar set of primitive application rules result in the {\em
truth values\/} \Code{#t}, \Code{#f}:
\begin{eqnarray*}
\Code{(< 6 7)}  & \rulesto & \Code{#t}\\
\Code{(not #f)}  & \rulesto & \Code{#t}\\
\Code{(zero? -1.414)}  & \rulesto & \Code{#f}\\
\vdots
\end{eqnarray*}
For example, applying these rules and the numerical ones above allows us
to handle:
\begin{eqnarray*}
\Code{(not (< (+ 0 1 2) (/ 3 1)))}  & \rulesto & \Code{(not (< (+ 0 1 2) 3))}\\
{}&  \rulesto & \Code{(not (< 3 3))}\\
{}&  \rulesto & \Code{(not #f)}\\
{}&  \rulesto & \Code{#t}
\end{eqnarray*}

It is easy to see that {\em it does not matter in what order the primitive
rewriting rules are applied}---the primitive value at the end will always
be the same!\footnote{Well, there's a {\em proviso\/} here: the primitive
value is unique when it is {\em possible} somehow to rewrite the
expression into one.  Some ``ill-typed'' expressions like \Code{(/ \#t 3)}
do not have values.  (What would Scheme do when such an expression is
evaluated?)} For example, we might have chosen to rewrite the example
above in another order:
\begin{eqnarray*}
\Code{(not (< (+ 0 1 2) (/ 3 1)))}  & \rulesto & \Code{(not (< 3 (/ 3 1)))}\\
{}&  \rulesto & \Code{(not (< 3 3))}\\
{}&  \rulesto & \Code{(not #f)}\\
{}&  \rulesto & \Code{#t}
\end{eqnarray*}
The primitive value is unique because Scheme is {\em fully
parenthesized}---ambiguous mathematical expressions like $(3 + 2 - 4)$ are
disallowed---and because there are no primitive rules with the same
lefthand side but different righthand sides.

\subsection{Conditional Rules}
There are two rewrite rules for an \Code{if} expression:
\begin{bnf}
\begin{eqnarray*}
\Code{(if #f \expr_{1} \expr_{2} )} & \rulesto & \expr_{2}\\
\Code{(if $<non-"#f" immediate value>$ \expr_{1} \expr_{2})} & \rulesto & \expr_{1}
\end{eqnarray*}
\end{bnf}

Note that, in contrast to primitive applications, the \Code{if} rule can
be applied without the constraint that $\expr_{1}$ and $\expr_{2}$ be
immediate values.

Now we can do rewriting such as
\begin{align*}
\Code{(if (lambda (a b) c) }
    &\Code{(if (not (< (+ 0 1 2) (/ 3 1))) atan (lambda (x) x)) +)}\\
{}\rulesto{} & \Code{(if (not (< (+ 0 1 2) (/ 3 1))) atan (lambda (x) x))}\\
\vdots\\
\rulesto{} & \Code{(if \#t atan (lambda (x) x))}\\
\rulesto{} & \Code{atan}
\end{align*}

\subsection{Instantiation}
The {\em variable instantiation rule\/} is:
\begin{quote}
\[\vrbl \rulesto \ival\]
when \vrbl\ is defined to be \ival.
\end{quote}

For example,
\begin{eqnarray*}
\lefteqn{\Code{(define y 7) (define x (+ 1 2)) (* x y)}}\\
{} & \rulesto & \Code{(define y 7) (define x (+ 1 2)) (* x 7)}\\
{} & \rulesto & \Code{(define y 7) (define x 3) (* x 7)}\\
{} & \rulesto & \Code{(define y 7) (define x 3) (* 3 7)}\\
{} & \rulesto & \Code{(define y 7) (define x 3) 21}
\end{eqnarray*}

Actually, to be used as a term rewriting rule at an arbitrary place in a
body, the instantiation rule must also include a technical specification
about changing defined names into ``fresh'' names\footnote{The problem is
that after instantiation, there are two copies of \ival---the original one
in the definition of \vrbl, and another at the place where \vrbl\ was
instantiated.  It may happen that some variable {\em within} \ival\ winds
up underneath a different definition in the second copy than it did in the
original.  This has to be prevented in order to mantain the requirement
that Scheme be statically scoped.}.  Fortunately, this technicality can
safely be ignored as long as we obey
\begin{quote}
{\bf The Lambda Body Rewriting Restriction}: {\em Don't instantiate into
the body of a lambda expression.}
\end{quote}
Real Scheme evaluation obeys the lambda body restriction.

\subsection{Lambda Applications and Renaming}

Scheme grammar allows definitions only in a sequence at the beginning of a
body.  Bodies can only appear by themselves at ``top-level,'' or as the
bodies of lambda expressions\footnote{A formal grammar for TRM Scheme is
attached as an Appendix for your reference.}.

When the lambda application rule creates new immediate
definitions\footnote{To obtain ``call-by-name,'' or ``normal order,''
behavior, we must relax the condition that the operands of the combination
be \ival's, allowing them instead to be arbitrary expressions.}, we just
stick them anyplace that is grammatical.  Definitions at the beginning of
a lambda body may also have to be moved after the lambda disappears.
So the form of the lambda application rule is:
\[\Code{((lambda (\vrbl_{1}\ldots) \defines\ \expr) \ival_{1}\ldots)} \rulesto
\expr\]
\begin{quote}
with the sequence
\[\Code{(define \vrbl_1 \ival_{1})}\ldots\defines\]
placed in the body above \expr\ at any position at which a sub-body may
begin.
\end{quote}
But watch out!  This time there is a naming problem we cannot avoid.

The scope, or binding, rules of a language determine how names are
associated with things named.  Scheme is a {\em statically} scoped
language (also called {\em lexically} scoped).  The purpose of static
scoping is to nsure that names of formals or names defined within an
expression can be ``kept private'' so they behave independently of names
outside the expression.  In this way, a programmer can choose names within
a program without concern for whether or how the same names may have been
used elsewhere.

For example, in Scheme, evaluation of the following body returns
the value 8, as we would expect:
\begin{lisp}
(define make-incrementer (lambda (n) (lambda (m) (+ m n))))
(define add2 (make-incrementer 2))
(define add3 (lambda (n) (inc (add2 n))))
(add3 5)
\end{lisp}

As the names suggest, \Code{add2} has the behavior of adding two to its
operand; similarly, \Code{add3} adds three.  Let's examine in detail how
our Substitution Model realizes this intended behavior.

The only rule applicable at the start is instantiatiation for
\Code{make-incrementer} or \Code{add3}.  Scheme would work on the
definitions at the beginning first, so let's do likewise and first
instantiate \Code{make-incrementer}:
\begin{lisp}
(define make-incrementer (lambda (n) (lambda (m) (+ m n))))
(define add2 ((lambda (n) (lambda (m) (+ m n))) 2))
(define add3 (lambda (n) (inc (add2 n))))
(add3 5)
\end{lisp}
It will be easier to follow this example if we now simplify the body by
dropping the no longer referenced definition---in this case, the
definition of \Code{make-incrementer}:
\begin{lisp}
(define add2 ((lambda (n) (lambda (m) (+ m n))) 2))
(define add3 (lambda (n) (inc (add2 n))))
(add3 5)
\end{lisp}
Lambda application in the definition of \Code{add2} creates a definition
of \Code{n}:
\begin{lisp}
(define add2 (lambda (m) (+ m n)))
(define n 2)
(define add3 (lambda (n) (inc (add2 n))))
(add3 5)
\end{lisp}
Now we can instantiate \Code{add3} and then drop its definition:
\begin{lisp}
(define add2 (lambda (m) (+ m n)))
(define n 2)
((lambda (n) (inc (add2 n))) 5)
\end{lisp}
Lambda application comes next, yielding:
\begin{lisp}
(define add2 (lambda (m) (+ m n)))
(define n 2)
(define n 5)
(inc (add2 n))
\end{lisp}

Now there is an obvious problem, because we have two definitions of
\Code{n} at the beginning of the block, which is ambiguous and indeed
violates Scheme grammar.  We handle this by renaming the newly defined
\Code{n} and the occurrences it defines, to some ``fresh'' name, say
\Code{n#1}:
\begin{lisp}
(define add2 (lambda (m) (+ m n)))
(define n 2)
(define n#1 5)
(inc (add2 n#1))
\end{lisp}
This expression is grammatical again, and now it is safe to instantiate
and garbage-collect \Code{add2}:
\begin{lisp}
(define n 2)
(define n#1 5)
(inc ((lambda (m) (+ m n)) n#1))
\end{lisp}

Notice that if we had not renamed \Code{n} to \Code{n#1} in this way, we
would have rewritten to
\begin{lisp}
(define n 2)
(define n 5)
(inc ((lambda (m) (+ m n)) n))
\end{lisp}
and the ambiguity of which occurrence of \Code{n} is bound to which value
becomes unresolvable\footnote{Actually, until the mid-1970's, all LISP
dialects resolved the ambiguity by choosing the ``nearest'' value of
\Code{n}, so that this example would have rewritten to \Code{(inc ((lambda
(m) (+ m 5)) 5))} and resulted in the value 11 instead of 8.  This method
of name management is called ``dynamic scope.''}.

So a more accurate statement of the {\em lambda application rule} is:
\[\Code{((lambda (\vrbl_{1} \ldots) \defines \expr) \ival_{1} \ldots)} \rulesto
\expr\]
\begin{quote}
with the sequence
\[\Code{(define \vrbl_1 \ival_{1})}  \ldots \defines\]
placed above \expr\ in any position in the body at which a sub-body may
begin, {\bf with defined variables freshly renamed as necessary}.
\end{quote}

\subsection{Garbage Collection}

``Garbage collection'' refers to the process whereby LISP-like systems
identify and recapture previously used, but no longer accessible, storage
cells.  The corresponding process in our TRM is dropping
unreferenced definitions, so we call this the {\em garbage collection
rule}:
\begin{eqnarray*}
\bod_1 & \rulesto& \bod_2,\\
\mbox{where $\bod_2$ is the } & \mbox{result} & \mbox{ of erasing garbage in $\bod_1$}.
\end{eqnarray*}
A subset $G$ of immediate definitions among the definitions occurring at
the beginning of a body is said to be {\em garbage in the body}, if none
of the variables left in the body after $G$ is erased were defined by $G$.

For example, in
\begin{lisp}
(define a (lambda () b))
(define b 3)
(define c (lambda () (* b d)))
(define d 4)
(+ 1 (a))
\end{lisp}
The definitions of \Code{c} and \Code{d} are {\em garbage} because
\Code{c}'s and \Code{d}'s don't occur anywhere outside their own
definitions.  The body can be garbage collected to be:
\begin{lisp}
(define a (lambda () b))
(define b 3)
(+ 1 (a))
\end{lisp}


\section{Uniqueness}

The full set of term rewriting rules has same the unique value
property as the simple algebraic rules:
\begin{quote}
\noindent{\bf The Unique Value Theorem}.  {\em There is at most one
primitive value that can be reached by successively rewriting a body using
the Scheme Term Rewriting rules above.}
\end{quote}

That's the good news.  The bad news is that the order in which rules
are applied {\em does} matter to avoid wasted time and space---even
infinite waste if rules are applied forever in useless places such as
the alternative branch of an \Code{if} whose test evaluates to
\Code{#t}.

We also observed that real Scheme evaluation obeys the lambda body
restriction.  There is no problem about this because of the
\begin{quote}
\noindent{\bf Lambda Body Safety Theorem}.  {\em If a body can be
rewritten into a primitive value by successive use of the Scheme Term
Rewriting rules above, then it can be rewritten to a primitive value while
obeying the lambda body rewriting restriction.}
\end{quote}

Of course by the Uniqueness Theorem, rewriting while obeying the lambda
body restriction will reach the {\em same} primitive value, if any, as any
other way of applying the rules.

The Uniqueness and Safety Theorems are by no means obvious.  (Their proof
is sometimes covered in the graduate course 6.840 on Semantics of
Programming Languages.)  There is a whole subdiscipline of Theoretical
Computer Science called Term Rewriting Theory which studies such
properties.

\section{Derived Expressions}

We have described above a ``kernel'' of Scheme which omits many familiar,
convenient Scheme constructs.  For example, we want to handle \Code{cond}
and \Code{let}.

Instead of extending the rewrite rules to handle these special forms
directly, another approach is to think of these extensions as
abbreviations, or ``syntactic sugar,'' for kernel expressions.  For
example,
\[
\Code{(let (($\vrbl_1$ $\init_1$)$\ldots\,$) \bod)}
\]
can be understood as an abbreviation for
\[
\Code{((lambda ($\vrbl_1\cdots\,$) \bod) $\init_1 \ldots\,$).}
\]
and
\begin{eqnarray*}
\Code{(cond } & \Code{($\test_{1}$ $\expr_{1}$)} \\*
              & \Code{($\test_{2}$ $\expr_{2}$)} \\*
              & \vdots \\*
              & \Code{(else $\expr_{n}$))}
\end{eqnarray*}
can be understood as an abbreviation for
\[
    \setlength{\arraycolsep}{0pt}
    \begin{array}{lllllllllllll}
    \Code{(if } & \rlap{\test_{1}} \\
                & \rlap{\expr_{1}} \\
                & \Code{(if } & \test_{2} \\
                &             & \rlap{\expr_{2}} \\
                &             & \vdots \\
                &             & \qquad \Code{(if } & \test_{n-1} \\
                &             &                    & \expr_{n-1} \\
                &             &                    & \expr_{n}\Code{)\ldots))}.
    \end{array}
\]
So we can specify the computations of derived expressions by
translating, or ``desugaring,'' them into kernel \expr's\footnote{These
translations are generally linear-time, one-pass, and yield a
kernel-\expr\ of size proportional to the original extended-\expr.  Real
Scheme interpreters and compilers typically carry out such translations.}.
The \RFS\ Report describes the translations above and others for \andsf's,
\orsf's, and several further forms.

\section{Lists and Symbols}

Scheme output notation for lists and symbols makes them look just like
expressions.  Using this notation in the TRM presents a problem: how do we
tell the difference between lists and combinations\footnote{The fact that
lists print out looking like expressions is generally considered a very
valuable feature in Lisp.  On the other hand, at least one famous
contemporary Computer Scientist has protested publicly that because of
this he couldn't learn Lisp---he could never figure out when to stop
evaluating expressions.}?  For example, is ``\Code{(+ 3 4)}'' a
combination---which should be rewritten to the syntactic value
\Code{7}---or is it a list of three elements---the procedure variable
\Code{+} (or perhaps it is the symbol ``\Code{+}'') followed by \Code{3}
and \Code{4}?  So the TRM instead ``simplifies'' lists into syntactic
values which are nested \Code{cons}'s of list elements.  For example, the
expression \Code{(list + 3 4)}, which Scheme prints out as
``\Code{([compiled arithmetic procedure +] 3 4)},'' would be rewritten in
our TRM into the expression
\[\Code{(cons + (cons 3 (cons 4 ())))}.\]
Similarly, {\em symbols\/} which print out as \Code{a} or \Code{+}
would be described in the TRM by the expressions \Code{(quote a)} or
\Code{(quote +)} to avoid confusing them with variables.  So
\Code{(list '+ 3 4)}, which Scheme prints out as ``\Code{(+ 3 4)},''
would would be rewritten in our TRM into the expression
\[\Code{(cons (quote +) (cons 3 (cons 4 ())))}.\]

With these notational conventions, it is a straightforward to handle
list structures in the TRM\footnote{This way of handling list works
  usefully until we begin ``mutating'' lists, at which point some more
  refined description of list structure would be needed.}.  Namely, we
add the appropriate procedure variables such as \Code{cons},
\Code{car} and \Code{cdr}, and regard any expression of the form
\[\bnf\Code{(cons \ival\ \ival)}\]
as an immediate value.  We don't even have to add a rewrite rule for
applications of \Code{cons}!  We do add some obvious rules for \Code{car}
and \Code{cdr}:
\begin{eqnarray*}
\Code{(car (cons \ival_{1} \ival_{2}))}
& \rulesto & \ival_{1} \\
\Code{(cdr (cons \ival_{1} \ival_{2}))}
& \rulesto & \ival_{2} \\
\end{eqnarray*}
The rules for \Code{cons}, \Code{car} and \Code{cdr} resemble the
primitive procedure rules in that they require that all the operands be
\ival's before the rule can be implied.  These {\em rule-specified
procedure variables} are, like primitive procedure variables, also treated
as \ival's.

Here are the remaining important rewriting rules about lists:
\begin{gather*}
\Code{(apply $E_{0}$ (cons $E_{1} \ldots$ (cons $E_{n}$ ())\ldots ))}
\rulesto \Code{($E_{0}\ E_{1}\ldots E_{n}$)}
\\[\bigskipamount]
\begin{aligned}
\Code{(pair? (cons \ival_{1} \ival_{2})} &\rulesto \Code{\#t}\\
\Code{(pair? \ival_{3})} &\rulesto \Code{\#f}\\
\end{aligned}\\
\text{if $\ival_{3}$ is not of
the form \Code{(cons \ival_{1} \ival_{2})}}
\\[\bigskipamount]
\begin{align*}
\Code{(null? ())}   & \rulesto \Code{\#t}\\
\Code{(null? \ival)} & \rulesto \Code{\#f}\\
\mbox{if \ival}  & \mbox{ is } \mbox{not \Code{()}.}
\end{align*}\\[\bigskipamount]
\begin{align*}
\Code{(list)} & \rulesto  \Code{()}\\
\Code{(list \ival_{1} \ldots)}  & \rulesto  \Code{(cons \ival_{1}
(list \ldots))}
\end{align*}
\end{gather*}


\section{Running the Substitution Model}

There is an implementation of the Substitution Model which you are welcome
to use.  It may help you understand how Scheme processes behave, and it
can also serve as a simple debugging aid.  Of course it is about four
orders of magnitude slower than Scheme, so don't expect it to be useful on
compute-intensive examples.

In the 6.001 Edwin editor, use {\tt M-x load-problem-set: 4} to load the
code for the Substitution Model.  You can generate a list of the steps in
the rewriting of an expression (in reverse order) by applying the
procedure \Code{smstep-list} to the quoted expression.  The order in which
rewrite rules are applied reflects Scheme evaluation order fairly
accurately.  To generate the list and print it nicely, apply the procedure
\Code{smeval} to the quoted expression.

For example, evaluate
\begin{lisp}
(smeval '((lambda (n) (+ 2 n)) 3))
\end{lisp}
to see expressions at selected steps as this lambda application
successively rewrote to \Code{5}.

A body beginning with one or more definitions is represented as a list of
the definitions with an expression at the end.  So evaluating,
\begin{lisp}
(smeval
 '((define (rec-factorial n)
     (if (<= n 0)
         1
         (* n (rec-factorial (dec n)))))
   (rec-factorial 6)))
\end{lisp}
will allow you to watch how this body rewrote to \Code{720}.

An expression printed out by \Code{smeval} at any point is a well-formed
Scheme expression which can be evaluated in Scheme and will give the same
final result as \Code{smeval}---at least if the final result is a number
or truth value\footnote{To evaluate a {\em body} printed by \Code{smeval}
when the body begins with definitions, it has to be made into a Scheme
expression.  If the body printed out is
\[\bnf \Code{(}\define_{1} \ldots \expr \Code{)}, \]
then in Scheme evaluate
\[\bnf \Code{((lambda () } \define_{1} \ldots \expr\Code{))}.\]}.


\subsection{Controlling Printout}

There are several utilities to control the output printed by \Code{smeval}.

The value returned by a call to \Code{smeval} is the \val\ at
which it successfully stops, or else an error message.  To be able to see
the final value in more familiar format, you should evaluate
\begin{lisp}
(define reversed-steps (smstep-list body))

(define final-body (body-or-info-of-step (car reversed-steps)))

(pp (printable-version final-body))
\end{lisp}

The procedure \Code{save-this-step?} determines which steps get saved on
the list generated by \Code{smstep-list} for printing by \Code{smeval}.
The default is to print more sparsely as the step-number grows.  To save
every step, evaluate
\begin{lisp}
(define (save-this-step? step-number body) #t)
\end{lisp}

The procedure \Code{garbage-collect-this-step?} can be used to force more
frequent garbage collections.  Its default arbitrarily imposes a garbage
collection every fortieth step.  To avoid such extra garbage collection
(which can slow things down a bit), evaluate
\begin{lisp}
(define (garbage-collect-this-step? step-number body) #f)
\end{lisp}

The procedure \Code{interrupt-smeval?} limits the number of rewriting
steps in an evaluation.  Its default is to abort evaluation after 600
steps.  You may want to do shorter runs.  For example, to set
\Code{smeval} to stop after 100 rewriting steps, evaluate
\begin{lisp}
(define (interrupt-smeval? step-number body) (> step-number 100))
\end{lisp}

{\bf Warnings}:
The code for the TRM implementation is written to be readable by
(ambitious) beginning 6.001 students, with major sacrifices in speed and
resilience for the sake of clarity.  As a consequence, \Code{smeval}
typically runs forever or crashes gracelessly on ungrammatical inputs.  It
is a good idea to test expressions by evaluating them in Scheme before
\Code{smeval}'ing them.

Edwin itself wedges horribly if the \Code{*scheme*} buffer overflows
(which may happen when its size is within a small factor of 100K
characters).  You don't want this to happen!  So be careful about doing
\Code{smeval} for long computations; instead save the list
\Code{smstep-list} generates and view it selectively.

\subsection{Bugs}

We don't know of any yet, but there surely are some.  Gratitude and lots
of 6.001 brownie points for first bug reports---email them to
\Code{6001-lecturers@ai.mit.edu}.

\section{Appendix: Grammar for Functional Scheme}

\subsection{Kernel Syntax}

\begin{bnf}
\begin{eqnarray*}
<define> &:=& "(define"\: <variable> <expression> ")"\\
<body>  &::=& <define_{1}>\ldots <expression>\\
        && {} \qquad \hbox{(Note: all defined variables must be distinct)}\\
\\
<expression> &::=& <variable> | <self-evaluating> | <symbol> | "()"\\
        && {}   | <combination> | <lambda expression> | <if>\\
\\
<self-evaluating>       &::=& <numeral> | <boolean>\\
<numeral>   &::=&   "0" | "-1" | "3.14159" | \ldots\\
<boolean>   &::=&   "#t" | "#f" \\
\\
<symbol>        &::=& "(quote " <variable> ")" | \Code{(quote $<keyword>$)}\\
\\
<combination>   &::=&   "(" <operator> <operand_{1}> \ldots ")"\\
<operator>   &::=&   <expression>\\
<operand>   &::=&   <expression>\\
\\
<lambda expression> &::=&  "(lambda ("<formals>") " <body> ")" \\
<formals>   &::=&   <variable_{1}>\ldots\\
        && {} \qquad \hbox{(Note: all \vrbl's must be distinct)}\\
\\
<if> &:=& "(if"\: <test> <consequent> <alternative> ")" \\
<test>   &::=&   <expression>\\
<consequent>   &::=&   <expression>\\
<alternative>   &::=&   <expression>\\
\\
<primitive procedure variable> &::=& "+" | "-" | "*" | "/" | "=" | "<" | "atan" | \dots\\
<rule-specified procedure variable> &::=& "null?" | "pair?" | "car" |
"cdr" | "cons" | "list" | "equal?" | "apply"\\
<keyword>   &::=& "define" | "quote" | "lambda" | "if"\\
<variable> &::=& \hbox{identifiers which are not $\bnf
        <self-evaluating>$ or $\bnf <keyword>$'s}\\
\end{eqnarray*}
\end{bnf}


\subsection{Derived Syntax}

\begin{bnf}
\begin{eqnarray*}
\expr &::=& \ldots | <derived expression>\\
<derived expression> &::=& <cond> | <let> | \ldots\\
<cond> &::=& "(cond"\: <clause_{1}> \ldots ")"\\
<clause> &::=& "(" <test> <expression>")" | "(else " \expr ")"\\
<let> &::=& "(let ("<binding_{1}> \ldots") " <body> ")" \\
<binding> &::=& "(" <variable> <init> ")"\\
<init> &::=& <expression>\\
%<begin> &::=& "(begin"\: \expr_{1} \ldots ")"\\
<define> &::=& \ \ldots\, | "(define ("<variable> <formals>")" <body>")"\\
%<primitive procedure variable> &::=& \ \ldots\, | "display" | "newline"\\
<keyword>  &::=&  \ \ldots\, | "cond" | "let" | "else" | \ldots\\
%"begin" | \ldots
\end{eqnarray*}
\end{bnf}


\subsection{Values}

\begin{bnf}
\begin{eqnarray*}
\pval &::=&  <self-evaluating> | <symbol> | "()"\\
\ival &::=& \pval | <lambda expression>\\
        && {} | <primitive procedure variable>\\
        && {} | <rule-specified procedure variable>\\
        && {} | "(cons "<immediate value>\ <immediate value>")"\\
<immediate define> &::=& "(define"\: <variable> <immediate value>")" \ldots\\
<value> &::=& <immediate define_{1}> \ldots <immediate value>\\
        && {} \qquad \hbox{(Note: all defined \vrbl's must be distinct)}\\
\end{eqnarray*}
\end{bnf}

\end{document}

\input ../6001mac

\begin{document}

\psetheader{Sample Problem Set}{Streams And Series}
\begin{center}
{\bf Streams}
\end{center}

\section{Part 1: Tutorial exercises}

Prepare the following exercises for oral presentation in tutorial:

\paragraph{Tutorial exercise 1:} 
Describe the streams produced by the following definitions.  Assume
that {\tt integers} is the stream of non-negative integers (starting
from 1):

\beginlisp 
(define A (cons-stream 1 (scale-stream 2 A)))
\null
(define (mul-streams a b)
  (cons-stream
   (* (stream-car a) (stream-car b))
   (mul-streams (stream-cdr a)
                (stream-cdr b))))
\null
(define B (cons-stream 1 (mul-stream B integers)))
\endlisp

\paragraph{Tutorial exercise 2} 

Given a stream {\tt s} the following procedure returns the stream of
all pairs of elements from {\tt s}:

\beginlisp
(define (stream-pairs s)
  (if (stream-null? s)
      the-empty-stream
      (stream-append
       (stream-map
        (lambda (sn) (list (stream-car s) sn))
        (stream-cdr s))
       (stream-pairs (stream-cdr s)))))
\null
(define (stream-append s1 s2)
  (if (stream-null? s1)
      s2
      (cons-stream (stream-car s1)
                   (stream-append (stream-cdr s1) s2))))
\endlisp

\noindent
(a) Suppose that {\tt integers} is the (finite) stream 1, 2, 3, 4, 5.
What is {\tt (stream-pairs s)}?  (b) Give the clearest explanation
that you can of how {\tt stream-pairs} works.  (c) Suppose that {\tt
s} is the stream of positive integers.  What are the first few
elements of {\tt (stream-pairs s)}?  Can you suggest a modification of
{\tt stream-pairs} that would be more appropriate in dealing with
infinite streams?


\section{Part 2: Laboratory---Using streams to represent power series}

We described in lecture a few weeks ago how to represent polynomials
as lists of terms.  In a similar way, we can work with {\it power
series}, such as
\begin{eqnarray*}
e^{x} &=&
1+x+\frac{x^{2}}{2}+\frac{x^{3}}{3\cdot2}+\frac{x^{4}}{4\cdot 3\cdot
2}+\cdots \cr
\cos x &=& 1-\frac{x^{2}}{2}+\frac{x^{4}}{4\cdot 3\cdot 2}-\cdots \cr
\sin x &=& x-\frac{x^{3}}{3\cdot 2}+\frac{x^{5}}{5\cdot 4\cdot 3\cdot 2}- \cdots
\end{eqnarray*}
represented as streams of infinitely many terms.  That is, the power
series

\[ a_0 + a_1 x + a_2 x^2 + a_3 x^3 + \cdots \]

\noindent
will be represented as the infinite stream whose elements are $a_0,
a_1, a_2, a_3, \ldots$.\footnote{In this representation, all streams are
infinite: a finite polynomial will be represented as a stream with an
infinite number of trailing zeroes.}

Why would we want such a method?  Well, let's separate the idea of a
series representation from the idea of evaluating a function.  For
example, suppose we let $f(x) = \sin x$.  We can separate the idea of
evaluating $f$, e.g., $f(0) = 0, f(.1) = 0.0998334$, from the means we
use to compute the value of $f$.  This is where the series
representation is used, as a way of storing information sufficient to
determine values of the function.  In particular, by substituting a
value for $x$ into the series, and computing more and more terms in
the sum, we get better and better estimates of the value of the
function for that argument.  This is shown in the table, where $\sin
{1 \over 10}$ is considered.

\begin{tabular}{lllll}
Coefficient&$x^n$&term&sum&value\\
\hline
0& 1& 0 & 0& 0\\
\\
1& ${1 \over 10}$& ${1 \over 10}$ & ${1 \over 10}$& .1\\
\\
0&${  1 \over 100}$& 0 &  ${1 \over 10}$& .1\\
\\
- ${1 \over 6}$& ${1 \over 1000}$& - ${1 \over 6000}$ & ${599 \over 6000}$& .099833333333\\
\\
0& ${1 \over 10000}$& 0 & ${599 \over 6000}$& .099833333333\\
\\
${1 \over 120}$& ${1 \over 100000}$& ${1 \over 12000000}$ & ${1198001 \over
12000000}$& .09983341666\\
\\
\end{tabular}

The first column shows the terms from the series representation for
sine.  This is the infinite series with which we will be dealing.  The
second column shows values for the associated powers of ${1 \over 10}$.
The third column is the product of the first two, and represents the
next term in the series evaluation.  The fourth column represents the
sum of the terms to that point, and the last column is the decimal
approximation to the sum.

With this representation of functions as streams of coefficients,
series operations such as addition and scaling (multiplying by a
constant) are identical to the basic stream operations.  We provide
series operations, though, in order to implement a complete power
series data abstraction:

\beginlisp
(define (add-streams s1 s2)
  (cond ((stream-null? s1) s2)
        ((stream-null? s2) s1)
        (else
         (cons-stream (+ (stream-car s1) (stream-car s2))
                      (add-streams (stream-cdr s1)
                                   (stream-cdr s2))))))
\null
(define (scale-stream c stream)
  (stream-map (lambda (x) (* x c)) stream))
\null
(define add-series add-streams)
\null
(define scale-series scale-stream)
\null
(define (negate-series s)
  (scale-series -1 s))
\null
(define (subtract-series s1 s2)
  (add-series s1 (negate-series s2)))
\endlisp

\noindent You can use the following procedure to examine the series you will
generate in this problem set:

\beginlisp
(define (show-series s nterms)
  (if (= nterms 0)
      'done
      (begin (write-line (stream-car s))
             (show-series (stream-cdr s) (- nterms 1)))))
\endlisp

\noindent
You can also examine an individual coefficient (of $x^n$) in a series
using {\tt series-coeff}:

\beginlisp
(define (series-coeff s n)
  (stream-ref s n))
\endlisp        

We also provide two ways to construct series.  {\tt Coeffs->series}
takes an list of initial coefficients and pads it with zeroes to
produce a power series.  For example, {\tt (coeff->series '(1 3 4))}
produces the power series $1+3x+4x^2+0x^3+0x^4+\ldots{}$.

\beginlisp
(define (coeffs->series list-of-coeffs)
  (define zeros (cons-stream 0 zeros))
  (define (iter list)
    (if (null? list)
        zeros
        (cons-stream (car list)
                     (iter (cdr list)))))
  (iter list-of-coeffs))
\endlisp

{\tt Proc->series} takes as argument a procedure $p$ of one numeric
argument and returns the series
\[  p(0) + p(1)x + p(2)x^2 + p(3)x^3 + \cdots  \]
The definition requires the stream {\tt non-neg-integers} to be the stream of
non-negative integers: $0,1,2,3,\ldots{}$\,.

\beginlisp
(define (proc->series proc)
  (stream-map proc non-neg-integers))
\endlisp

\medskip

\noindent
{\bf Note:} Loading the code for this problem set will change
Scheme's basic arithmetic operations {\tt +}, {\tt -}, {\tt *}, and
{\tt /} so that they will work with rational numbers.  For instance,
{\tt (/ 3 4)} will produce 3/4 rather than .75.  You'll find this
useful in doing the exercises below.


\paragraph{Lab exercise 1:}
Load the code for problem set 9.  To get some initial practice with
streams, write and turn in definitions for each of the following:

\begin{itemize}

\item {\tt ones}: the infinite stream of 1's.

\item {\tt non-neg-integers}:  the stream of integers, $1, 2, 3, 4,
\ldots$

\item {\tt alt-ones}:  the stream $1, -1, 1, -1, \ldots$

\item {\it zeros}:  the infinite stream of 0's.  Do this using {\tt alt-ones}.

\end{itemize}


Now, show how to define the series:
\begin{eqnarray*}
S_1 &=& 1 + x + x^2 + x^3 + \cdots \cr
S_2 &=& 1 + 2x + 3x^2 + 4x^3 + \cdots
\end{eqnarray*}
Turn in your definitions and a couple of coefficient printouts to
demonstrate that they work.


\paragraph{Lab exercise 2:}

Multiplying two series is a lot like multiplying two multi-digit numbers,
but starting with the left-most digit, instead of the right-most.

For example:

\begin{verbatim}

                11111
             x  12321
            _________

            11111
             22222
              33333
               22222
                11111
            ---------
            136898631

\end{verbatim}

Now imagine that there can be an infinite number of digits, i.e., each of 
these is a (possibly infinite) series.  (Remember that because each "digit" 
is in fact a term in the series, it can become arbitrarily large, without
carrying, as in ordinary multiplication.)

Using this idea, complete the definition of the following procedure,
which multiplies two series:
\beginlisp
(define (mul-series s1 s2)
  (cons-stream $\langle E_1 \rangle$
               (add-series $\langle E_2 \rangle$
                           $\langle E_3 \rangle$)))
\endlisp
To test your procedure, demonstrate that the product of $S_1$ (from
exercise 1) and $S_1$ is $S_2$.  What is the coefficient of
$x^{10}$ in the product of $S_2$ and $S_2$?  Turn in your definition of
{\tt mul-series}.  (Optional: Give a general formula for the
coefficient of $x^n$ in the product of $S_2$ and $S_2$.)

\subsection*{Inverting a power series}

Let $S$ be a power series whose constant term is 1.  We'll call such a
power series a ``unit power series.''  Suppose we want to find the {\em
inverse} of $S$, namely, the power series $X$ such that $S\cdot X= 1$.
To see how to do this, 
write $S=1+S_R$ where $S_R$ is the rest of $S$ after the constant
term.  Then we want to solve the equation $S \cdot X = 1$ for $S$ and
we can do this as follows: 

\begin{eqnarray*}
S \cdot X &=& 1 \cr
(1+S_R)\cdot X &=& 1 \cr
X + S_R \cdot X &=& 1 \cr
X &=& 1 - S_R \cdot X
\end{eqnarray*}

In other words, $X$ is the power series whose constant term is 1 and
whose rest is given by the negative of $S_R$ times $X$.

\paragraph{Lab exercise 3:} Use this idea to write a procedure {\tt
invert-unit-series} that computes $1/S$ for a unit power series $S$.
To test your procedure, invert the series $S_1$ (from exercise 1) and
show that you get the series $1-x$.  (Convince yourself that this is
the correct answer.)  Turn in a listing of your procedure.  This is a
very short procedure, but it is very clever.  In fact, to someone
looking at it for the first time, it may seem that it can't
work---that it must go into an infinite loop.  Write a few sentences
of explanation explaining why the procedure does in fact work, and
does not go into a loop.

\paragraph{Lab exercise 4:} Use your answer from exercise 3 to produce
a procedure {\tt div-series} that divides two power series.  {\tt
Div-series} should work for any two series, provided that the
denominator series begins with a non-zero constant term.  (If the
denominator has a zero constant term, then {\tt div-series} should
signal an error.)  Turn in a listing of your procedure along with
three or four well-chosen test cases (and demonstrate why the answers
given by your division are indeed the correct answers).

\paragraph{Lab exercise 5:}  Now suppose that we want to integrate a
series representation.  By this, we mean that we want to perform
symbolic integration, thus, for example, 
given a series
\[ a_0 + a_1 x + a_2 x^2 + a_3 x^3 + \cdots \]
we want to return the integral of the series (except for the constant term) 
\[ a_0 x + \frac{1}{2}a_1 x^2 + \frac{1}{3}a_2 x^3 + \frac{1}{4}a_3 x^4 + \cdots \]

Define a procedure {\tt
integrate-series-tail} that will do this.   Note that all you need to
do is transform the series

\[ a_0 \ \ \ \ \  a_1 \ \ \ \ \  a_2  \ \ \ \ \  a_3  \ \ \ \ \   a_4  \ \ \ \ \   a_5  \ \ \ \ \  \cdots \]

into the series 

\[ a_0 \ \ \ \ \  {a_1 \over 2} \ \ \ \ \  {a_2 \over 3}  \ \ \ \ \
{a_3 \over 4}  \ \ \ \ \  
{a_4 \over 5}  \ \ \ \ \  
{a_5 \over 6}  \ \ \ \ \  \cdots \]

Note that this means that the procedure generates the coefficients of
a series starting with the first order coefficient, not that the
zeroth order coefficient is 0.

Turn in a listing of your procedure and demonstrate that it works by
computing {\tt integrate-series-tail} of the series $S_2$ from
exercise 1.

\paragraph{Lab exercise 6:} Demonstrate that you can generate the
series for $e^x$ as

\beginlisp
(define exp-series
  (cons-stream 1 (integrate-series-tail exp-series)))
\endlisp

\noindent Explain the reasoning behind this definition.  Show how to generate
the series for sine and cosine, in a similar way, as a pair of mutually
recursive definitions.  It may help to recall that the integral 
\[\int \sin x =  - \cos x\]
and that the integral
\[\int \cos x = \sin x\]


\paragraph{Lab exercise 7:} Louis Reasoner is unhappy with the idea
of using {\tt integrate-series-tail} separately.  ``After all,'' he
says, ``if we know what the constant term of the integral is supposed
to be, we should just be able to incorporate that into a procedure.''
Louis consequently writes the following procedure, using {\tt
integrate-series-tail}:

\beginlisp
(define (integrate-series series constant-term)
  (cons-stream constant-term (integrate-series-tail series)))
\endlisp

\noindent
He would prefer to define the exponential series as

\beginlisp 
(define exp-series
  (integrate-series exp-series 1))
\endlisp 

\noindent
Write a two or three sentence clear explanation of why this won't
work, while the definition in exercise 6 does work.

\paragraph{Lab exercise 8:} Write a procedure that produces the
derivative of a power series.  Turn in a definition of your procedure
and some examples demonstrating that it works.

\paragraph{Lab exercise 9:} Generate the power series for tangent,
and secant.  List the first ten or so coefficients of each
series.  Demonstrate that the derivative of the tangent is the square
of the secant.

\paragraph{Lab exercise 10:}  We can also generate power series for
inverse trigonometric functions.  For example:
\[\tan^{-1}(x) = \int_0^x {dz \over {1 + z^2}}\]
Use this equation, plus methods that you have already created, to
generate a power series for arctan. 
Note that $1+z^2$ can be viewed as a finite series.
Turn in your definition, and a
printout of the first few coefficients of the series.

\paragraph{Lab exercise 11:}  One very useful feature of a power
series representation for a function is that one can use the initial
terms in the series to get approximations to the function.  For
example, suppose we have

\[f(x) = a_0 + a_1x + a_2 x^2 + a_3 x^3 + \ldots\]

We have represented this by the series of coefficients:

\[\left\{a_0\ \ a_1\ \ a_2 \ \ a_3 \ \ \ldots\right\}\]

Now suppose that we want to approximate the value of the function $f$
at some point $x_0$.  We could successively improve this
approximation\footnote{actually there are some technical issues
about whether the argument is in the radius of convergence of the
series, but we'll ignore that issue here} by considering the following

\begin{eqnarray} f(x_0) \approx a_0\\ f(x_0) \approx a_0 + a_1 x_0\\
f(x_0) \approx a_0 + a_1 x_0 + a_2 x_0^2\\
\end{eqnarray}

Notice that each of these expressions (1), (2), (3) could also be captured in a
stream representation, with first term $a_0$, second term $a_0 + a_1
x_0$ and so on.

Implement this idea by defining a procedure {\tt approximate} which
takes as arguments a value $x_0$ and a series representation of a
function $f$, and which returns a stream of successive approximations
to the value of the function at that point $f(x_0)$.  Turn in a
listing
of your code, as well as examples of using it to approximate some
functions.
Note that to be very careful, this is not a series representation but
a stream one, so you may want to think carefully about which
representations to use.

{\bf Optional additional exercises}

he {\em Bernoulli numbers} $B_n$ are
defined by the coefficients in the following power series:%
%
\footnote{The Bernoulli numbers arise in a wide variety of
applications involving power series and approximations.  They were
introduced by Jacob Bernoulli in 1713.}%
%
\[
\frac{x}{e^x-1} = B_0 + B_1 x + \frac{B_2 x^2}{2!} + \frac{B_3
x^3}{3!} + \cdots = \sum_{k \geq 0} \frac{B_k x^k}{k!}
\]
Write a procedure that takes $n$ as an argument and produces the $n$th
Bernoulli number.  (Check: All the odd Bernoulli numbers are zero
except for $B_1=-1/2$.)  Generate a table of the first 10 (non-zero)
Bernoulli numbers.  Note: Since $e^x-1$ has a zero constant term, you
can't just use {\tt div-series} to divide $x$ by this directly.

It turns out that the coefficient of
$x^{2n-1}$ in the series expansion of $\tan x$ is given by
\[
\frac{(-1)^{n-1} 2^{2n} (2^{2n} - 1) B_{2n}}{(2n)!}
\]
Use your series expansion of tangent to verify this for a few of
values of $n$.


\end{document}
